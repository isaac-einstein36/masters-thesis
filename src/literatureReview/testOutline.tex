\section{Current Tumor Bed Localization Devices and Methods\label{sec:literatureReview:currentMethods}}
% Summarize current clinical methods for localizing the tumor bed post-lumpectomy.
% Include fiducial markers, seroma tracking, and implantable devices (e.g., VeraForm).
% Discuss key limitations such as migration, resorption, visibility on imaging, and biocompatibility issues.

\subsection{Challenges with Current Devices\label{sec:literatureReview:currentMethods:challenges}}
% Highlight why new localization methods are being explored.
% Cover limitations in accuracy, ease of placement, imaging visibility, and patient comfort.

\section{Extrusion Processes and Parameters\label{sec:literatureReview:extrusion}}
% Introduce extrusion in polymer processing.
% Emphasize its role in producing filaments for 3D printing and research applications.

\subsection{Tabletop Extrusion Mechanics\label{sec:literatureReview:extrusion:howWorks}}
% Describe barrel, screw, hopper, nozzle, and drive systems.
% Discuss typical lab-scale extruders (e.g., 3Devo, Filabot).

\subsubsection{Important Parameters\label{sec:literatureReview:extrusion:parameters}}
% Detail process parameters:
% heat zones, material size/uniformity, screw RPM, cooling rate, pre-drying requirements, and melt flow index.

\subsection{Extruding Regrind\label{sec:literatureReview:extrusion:regrind}}
% Cover reprocessing of polymer waste or failed prints.
% Mention equipment such as industrial grinders or Felfil Shredder.

\subsubsection{Effects of Re-Extruding Materials\label{sec:literatureReview:extrusion:regrindEffects}}
% Discuss degradation, viscosity changes, and mechanical property shifts.

\subsection{Powder Extrusions\label{sec:literatureReview:extrusion:powder}}
% Discuss challenges like clumping, inconsistent back pressure, and flow rate variability.

\subsection{Common Extrusion Issues\label{sec:literatureReview:extrusion:issues}}
% Detail potential issues:
% material clumping at the hopper, nozzle clogging, or uneven extrusion.

\subsection{Purging an Extruder\label{sec:literatureReview:extrusion:purging}}
% Explain importance of routine purging to maintain extrusion quality and prevent contamination.

\subsubsection{Importance of Purging\label{sec:literatureReview:extrusion:purgingImportance}}
% Note recommendations such as 3Devo’s monthly purging schedule or when switching to powder materials.

\subsubsection{Purging Procedures\label{sec:literatureReview:extrusion:purgingProcedures}}
% Summarize cleaning methods from vendors:
% - Devoclean MT
% - Dyna-Purge compounds
% Include stepwise summary of manufacturer-recommended methods.

\subsubsection{Purging Compounds\label{sec:literatureReview:extrusion:purgingCompounds}}
% Discuss vendor options and selection criteria.

\paragraph*{Vendors}
% Devoclean MT, Dyna-Purge L (soft), K (soft but abrasive), and D2 (hard and abrasive).

\paragraph*{Material Compatibility}
% Discuss melt flow index and temperature matching, and use of harder materials (e.g., HDPE) for flushing.

\section{3D Printing Processes\label{sec:literatureReview:3dPrinting}}
% Introduce additive manufacturing context for polymer research.

\subsection{Fused Deposition Modeling (FDM)\label{sec:literatureReview:3dPrinting:FDM}}
% Describe how FDM works, its typical applications, and its advantages over other AM methods.

\subsubsection{Key Parameters\label{sec:literatureReview:3dPrinting:parameters}}
% Discuss bed temperature, nozzle temperature, print speeds, and nozzle size/material.

\subsubsection{Optimal Printing Parameters\label{sec:literatureReview:3dPrinting:optimalParameters}}
% Include PLA, PCL, and PLCL printing settings.
% Discuss calibration methods (e.g., Prusa or Bambu calibration procedures).

\section{Creating PLCL Blends\label{sec:literatureReview:PLCL}}
% Introduce PLCL copolymer—properties, rationale, and optimal ratios.

\subsection{Optimal Composition\label{sec:literatureReview:PLCL:composition}}
% Note that 70/30 (PLA/PCL) is commonly used for balance of strength and flexibility.

\subsection{Synthesis Methods\label{sec:literatureReview:PLCL:synthesis}}
% Compare ring-opening polymerization (ROP) vs. block copolymer synthesis.

\subsection{Material Blending Techniques\label{sec:literatureReview:PLCL:blending}}
% Discuss melting (thermal fusion), solvent-based blending (chloroform or DCM), and injection molding.

\subsubsection{Injection Molding\label{sec:literatureReview:PLCL:injectionMolding}}
% Cover how injection molding promotes uniform mixing and compare with extrusion.

\subsubsection{Extruding Raw Materials Together\label{sec:literatureReview:PLCL:coextrusion}}
% Explain single-screw vs twin-screw extrusion advantages for homogeneity.

\section{Alternative Device Fabrication Approaches\label{sec:literatureReview:alternatives}}
% Review unconventional or hybrid manufacturing techniques.

\subsection{Composite Filaments\label{sec:literatureReview:alternatives:composite}}
% Summarize existing commercial composite filaments and their limitations.

\subsection{Embedded and Multi-Material Printing\label{sec:literatureReview:alternatives:multiMaterial}}
% Discuss embedded printing and multi-material printing techniques.
% See this paper: \url{https://www.sciencedirect.com/science/article/pii/S1120179722020518}

\section{Radiopaque Agents\label{sec:literatureReview:radiopacity}}
% Explain purpose of radiopaque agents for imaging visibility.

\subsection{Existing Radiopaque Filaments\label{sec:literatureReview:radiopacity:filaments}}
% Summarize known formulations and radiopaque additives (see Radiopaque Materials section in RefWorks).

\section{Imaging Studies\label{sec:literatureReview:imaging}}
% Review CT-based imaging techniques for quantifying radiopacity.

\subsection{Hounsfield Units (HU) Interpretation\label{sec:literatureReview:imaging:HU}}
% Discuss how HU relates to material density and visibility in CT.

\subsection{Test Standards\label{sec:literatureReview:imaging:standards}}
% Reference ASTM F640-20 and other applicable imaging test standards.

\subsection{Prior Studies\label{sec:literatureReview:imaging:priorStudies}}
% Include references such as Ref ID 77.

\section{Material Properties\label{sec:literatureReview:materialProperties}}
% Summarize mechanical and physical properties of relevant biological and synthetic materials.

\subsection{Breast Tissue Properties\label{sec:literatureReview:materialProperties:breast}}
% Include data for adipose and glandular tissue.

\subsection{Polymer Properties\label{sec:literatureReview:materialProperties:polymers}}
% Compare PLA, PCL, and PLCL baseline properties.

\subsection{Effects of Radiopaque Additives\label{sec:literatureReview:materialProperties:BaSO4}}
% Include findings from RefWorks on BaSO₄ incorporation and its effects on modulus and density.

\section{Mechanical Testing\label{sec:literatureReview:mechanicalTesting}}
% Describe standards and rationale for evaluating printed or molded specimens.

\subsection{Testing Standards\label{sec:literatureReview:mechanicalTesting:standards}}
% Include ASTM D638 (tensile) and ASTM D790 (flexural).

\subsection{Specimen Development\label{sec:literatureReview:mechanicalTesting:specimens}}
% Note use of Type-V specimens (ASTM D638) due to limited material.
% Discuss infill effects; 100\% infill used for consistency.

\subsection{Interpreting Mechanical Testing\label{sec:literatureReview:mechanicalTesting:interpretation}}
% Explain what tensile strength, elastic modulus, and flexural modulus reveal about material behavior.

\section{Characterizing Polymer Blends\label{sec:literatureReview:characterization}}
% Review analytical methods for confirming blend composition and structure.

\subsection{NMR Testing\label{sec:literatureReview:characterization:NMR}}
% Detail methods for determining % composition via NMR.

\subsubsection{Powder Preparation\label{sec:literatureReview:characterization:powderPrep}}
% Compare mortar and pestle, ball milling, and grinding filament options.

\subsection{Microscopy\label{sec:literatureReview:characterization:microscopy}}
% Mention SEM or optical microscopy for morphology and dispersion analysis.

\subsection{Gel Permeation Chromatography (GPC)\label{sec:literatureReview:characterization:GPC}}
% Discuss methods for molecular weight characterization.

\section{Project Road Mapping\label{sec:literatureReview:roadmapping}}
% Explore literature on project planning tools in research.

\subsection{Gantt Charts\label{sec:literatureReview:roadmapping:gantt}}
% Discuss visualization and time management benefits.

\subsection{Other Roadmapping Tools\label{sec:literatureReview:roadmapping:otherTools}}
% Mention alternative visual management tools and their impact on team coordination.

\section{Belt Tensioning Systems\label{sec:literatureReview:beltTension}}
% Review mechanical design considerations for belt-driven systems.

\subsection{Idler Pulley Placement\label{sec:literatureReview:beltTension:idler}}
% Compare inside vs outside idler pulley configurations.
% Reference: \url{https://www.barbieribelt.com/news/commonly-used-tensioning-methods-for-timing-belts}
