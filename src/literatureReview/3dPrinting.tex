\section{3D Printing Processes\label{sec:literatureReview:printing}}
% Introduce additive manufacturing context for polymer device fabrication.
Three-dimensional (3D) printing is a growing form of additive manufacturing. There are three general forms of manufacturing: additive such as 3D printing, subtractive such as milling, and formative such as injection molding~\cite{RefWorks:RefID:411-3d}. These three forms are illustrated below in Figure~\ref{fig:literatureReview:formsOfManufacturing}.

\begin{figure}[h!]
        \centering
        \includegraphics[width=\linewidth]{../figs/literature_review/3dPrinting/general_forms_of_manufacturing.png}
        \caption{Three general forms of manufacturing~\cite{RefWorks:RefID:411-3d}.}
        \label{fig:literatureReview:formsOfManufacturing}
\end{figure}

3D printing has become increasingly popular due to its ability to create complex shapes and with less waste than conventional manufacturing~\cite{RefWorks:RefID:409-kristiawan2021review}. 3D printing was initially only used for rapid prototyping but is now suitable for production of parts~\cite{RefWorks:RefID:411-3d}.

\subsection{Fused Deposition Modeling (FDM)\label{sec:literatureReview:printing:FDM}}
% Describe the principle of FDM, typical materials, and its advantages over other AM methods.

There are numerous forms of 3D printing processes including: vat polymerization, material extrusion, powder bed fusion, material jetting, binder jetting, direct energy deposition, and sheet lamination~\cite{RefWorks:RefID:411-3d}.

Fused Deposition Modeling (FDM), a form of material extrusion, is the most widely used 3D printing technology~\cite{RefWorks:RefID:411-3d}. FDM is fast, inexpensive, and can utilize common thermoplastics. However, FDM leaves a rough surface finish, often requires supports, and is difficult to scale~\cite{RefWorks:RefID:411-3d}.

FDM is performed by pulling a raw material, filament, through drive wheels and into a nozzle. This filament is then melted within the nozzle and deposited on a print bed as a semiliquid in ultrathin layers. The contours of this layer-by-layer structure are created by Computer Aided Design (CAD) modeling software~\cite{RefWorks:RefID:409-kristiawan2021review,RefWorks:RefID:411-3d}. A visualization of FDM printing is shown below in Figure~\ref{fig:literatureReview:fdmPrintingOverview}.

\begin{figure}[h!]
        \centering
        \includegraphics[width=0.7\linewidth]{../figs/literature_review/3dPrinting/fdm_printing_overview.png}
        \caption{Visualization of Fused Deposition Modeling (FDM) printing~\cite{RefWorks:RefID:307-cano-vicent2021fused}.}
        \label{fig:literatureReview:fdmPrintingOverview}
\end{figure}

Typical materials used in FDM printing are filaments composed of polylactic acid (PLA), acrylonitrile blends and acrylonitrile butadiene styrene (ABS), Nylon 12, and polypropylene (PP)~\cite{RefWorks:RefID:409-kristiawan2021review,RefWorks:RefID:411-3d}.

\subsection{Important Parameters\label{sec:literatureReview:printing:parameters}}
% Discuss bed temperature, nozzle temperature, print speeds, and nozzle size/material.

FDM printing involves multiple parameters that all need to be optimized for a specific filament type and filament size. These parameters are linked to one another and collectively affect the quality of the printed part~\cite{RefWorks:RefID:409-kristiawan2021review}.

These parameters can be divided into FDM machine parameters and working parameters~\cite{RefWorks:RefID:409-kristiawan2021review}.

\paragraph*{Machine Parameters}

Machine parameters include bed temperature, nozzle temperature, and nozzle diameter.

Bed temperature and nozzle temperature often fall under wide ranges and are dependent on the material being printed~\cite{RefWorks:RefID:307-cano-vicent2021fused}.

Nozzle diameter needs to be decided before printing a material. This parameter, based solely on the nozzle size installed, affects the detail of a print, the speed of the print, strength of the print, and risks of clogging. Smaller nozzles have better detail or resolution but slower print speeds while larger nozzles print quickly, help reduce clogging, and provide increased part strength. Common nozzle diameters in millimeters are 0.2, 0.4, 0.6, and 0.8~\cite{RefWorks:RefID:413-20253d}.

\paragraph*{Working Parameters}

Working parameters include factors such as raster angle, raster width, and build orientation. These are usually set internally in the slicing software when creating the printable file~\cite{RefWorks:RefID:409-kristiawan2021review}. Some of these working parameters are illustrated in Figure~\ref{fig:literatureReview:printingWorkingParamaters}.

\begin{figure}[h!]
        \centering
        \includegraphics[width=0.7\linewidth]{../figs/literature_review/3dPrinting/printing_working_parameters.png}
        \caption{Printing Working Parameters. (a) Build orientation, (b) layer thickness, and (c) tool path parameters~\cite{RefWorks:RefID:409-kristiawan2021review}.}
        \label{fig:literatureReview:printingWorkingParamaters}
\end{figure}

\subsection{Optimal Printing Parameters\label{sec:literatureReview:printing:optimalParameters}}

\subsubsection{Printing Temperature\label{sec:literatureReview:printing:optimalParameters:temperature}}

Printing parameters need to be optimized for each material used. Well-researched materials have pre-established parameters that can act as guidelines~\cite{RefWorks:RefID:307-cano-vicent2021fused}.

Optimal starting printing temperatures for materials used in this research, PLA, PCL, and PLCL, are provided in Table~\ref{tab:literatureReview:optimal_material_printing_parameters} as a reference point, though any new vendor or custom filament requires specific calibration.

\begin{table}[h!]
        \centering
        \caption{Extrusion and thermal properties of relevant polymer materials.}
        \label{tab:literatureReview:optimal_material_printing_parameters}
        \resizebox{\textwidth}{!}{%
                \begin{tabular}{lccc}
                        \hline
                        \textbf{Material}                                                  & \textbf{Extrusion Temperature (ºC)} & \textbf{Bed Temperature (ºC)} & \textbf{Glass Transition Temperature (ºC)} \\
                        \hline
                        PLA~\cite{RefWorks:RefID:307-cano-vicent2021fused}                 & 160–230                             & 20–60                         & 60–65                                      \\
                        PCL~\cite{RefWorks:RefID:408-tack20163dprinting}                   & 115–145                             & 30–45                         & -60                                        \\
                        PLCL~\cite{RefWorks:RefID:18-v2022assessing,RefWorks:RefID:204-70} & 180–230                             & 18–30                         & 35–40                                      \\
                        \hline
                \end{tabular}%
        }
\end{table}

\subsubsection{Printing Infill Density\label{sec:literatureReview:printing:optimalParameters:infillDensity}}

Infill density is the percentage of filament compared to empty space within a 3D printed part. As infill percentage, also referred to as infill density or occupancy rate, increases, internal air gaps decrease. This is illustrated below in Figure~\ref{fig:literatureReview:infillDensityExplained} helps illustrate what occurs internally as infill density changes.

\begin{figure}[h!]
        \centering
        \includegraphics[width=0.7\linewidth]{../figs/literature_review/3dPrinting/infill_density_explained.png}
        \caption{Visualization of changes in infill density~\cite{RefWorks:RefID:33-johnson2018evaluation}.}
        \label{fig:literatureReview:infillDensityExplained}
\end{figure}

Multiple studies have been conducted evaluating the effects of infill density on part strength. Overall, tensile strength increases as infill density increases, and many studies recommended a rectilinear infill with at least 50\% infill density for optimal strength~\cite{RefWorks:RefID:425-atakok2022tensile,RefWorks:RefID:426-kam2023investigation,RefWorks:RefID:429-rismalia2019infill,RefWorks:RefID:430-2016effect}.

However, studies found marginal returns printing above 75\% infill given the significantly increased time and materials to print at 100\% infill density~\cite{RefWorks:RefID:33-johnson2018evaluation,RefWorks:RefID:429-rismalia2019infill}. Tensile strength increased more substantially between 20 and 50\%~\cite{RefWorks:RefID:430-2016effect}.

If material properties are the largest interest rather than optimizing strength, 100\% infill is likely the optimal choice. It was found that using a 100\% infill most closely resembled the properties of the raw material~\cite{RefWorks:RefID:430-2016effect}.

\subsubsection{Build Plate Orientation\label{sec:literatureReview:printing:optimalParameters:orientation}}

Print orientation or how a part is positioned on the print bed plays an important role in the overall mechanical properties of the 3D printed part. Individual print layers are held together by interlocking print lines, but the layers themselves are only connected by adhesion among layers. As a result, individual layers provide strength but it takes less force to separate entire layers from one another~\cite{RefWorks:RefID:432-hanon2020effect}.

Because of this weakness between layer lines, the direction layers are placed with respect to the direction of force the part will experience should be considered. For example, for a uniaxial tensile test, it was found that printing a part on its edge rather than flat on the bed provided optimal strength due to the orientation of the print layers~\cite{RefWorks:RefID:432-hanon2020effect,RefWorks:RefID:427-karad2023experimental}. Figure~\ref{fig:literatureReview:printOrientation} illustrates the effect of print orientation relative to the forces a part may experience.

\begin{figure}[h!]
        \centering
        \includegraphics[width=0.7\linewidth]{../figs/literature_review/3dPrinting/print_orientation.png}
        \caption{Effects of print orientation on layer adhesion~\cite{RefWorks:RefID:432-hanon2020effect}.}
        \label{fig:literatureReview:infillDensitprintOrientationyExplained}
\end{figure}

\subsection{Calibrating Materials\label{sec:literatureReview:printing:calibratingMaterials}}
% Discuss calibration approaches such as for Prusa or Bambu to determine optimal print parameters
% Line test, calibration cube, temp tower, Bambu inherent calibrations

For standard or custom materials, running printer calibrations for each material will ensure consistent prints. First layer metrics, motor speeds, axes, and flow dynamics are all important aspects to be calibrated~\cite{RefWorks:RefID:414-halford2024calibrate}.

Most printer manufacturers have their own recommended calibration procedures, with a range of manual and automated procedures~\cite{RefWorks:RefID:415-flow,RefWorks:RefID:416-flow,RefWorks:RefID:417-2025layer}.

\paragraph*{First Layer Calibration}
The first layer calibration is used to calculate the optimal distance between the nozzle and print bed on the first layer. This ensures the material extruding from the nozzle adheres to the bed well and provides an adequate foundation for subsequent layers~\cite{RefWorks:RefID:414-halford2024calibrate,RefWorks:RefID:417-2025layer}. Figure~\ref{fig:literatureReview:effectsOfFirstLayerCalibration} shows the effects of a proper first layer calibration on the final printed part.

\begin{figure}[h!]
        \centering
        \includegraphics[width=0.7\linewidth]{../figs/literature_review/3dPrinting/effects_of_first_layer_calibration.png}
        \caption{Effects of proper first layer calibration~\cite{RefWorks:RefID:417-2025layer}.}
        \label{fig:literatureReview:effectsOfFirstLayerCalibration}
\end{figure}

\paragraph*{Flow Dynamics Calibration}
Flow dynamics are also an important parameter to adjust to help account for changes and lag in extrusion pressure from the nozzle~\cite{RefWorks:RefID:415-flow}. Some printers, such as Bambu A1 Series, can perform this calibration entirely automatically. Figure~\ref{fig:literatureReview:effectsOfFlowDynamics} highlights the importance of calibrating flow dynamics.

\begin{figure}[h!]
        \centering
        \includegraphics[width=0.9\linewidth]{../figs/literature_review/3dPrinting/effects_of_flow_dynamics_calibration.png}
        \caption{Effects of proper flow dynamics calibration~\cite{RefWorks:RefID:415-flow}.}
        \label{fig:literatureReview:effectsOfFlowDynamics}
\end{figure}

\paragraph*{Flow Rate Calibration}
Flow rate is also a common 3D printing calibration factor. This determines how much filament the nozzle pushes out or extrudes while printing. A high flow rate will lead to overextrusion, seen by blobs or overlapping lines. With a low flow, rate underextrusion can occur, seen by gaps in the print~\cite{RefWorks:RefID:416-flow}. Figure~\ref{fig:literatureReview:effectsOfFlowRate} shows the importance of calibrating the flow rate of a material.

\begin{figure}[h!]
        \centering
        \includegraphics[width=0.9\linewidth]{../figs/literature_review/3dPrinting/effects_of_flow_rate_calibration.png}
        \caption{Effects of proper flow rate calibration~\cite{RefWorks:RefID:416-flow}.}
        \label{fig:literatureReview:effectsOfFlowRate}
\end{figure}
