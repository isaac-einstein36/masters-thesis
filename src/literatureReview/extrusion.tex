\section{Extrusion Processes and Parameters\label{sec:literatureReview:extrusion}}
% Introduce extrusion in polymer processing and its relevance to filament production.
Extrusion is a form of additive manufacturing (AM) in which material is melted and dispensed through a nozzle or orifice. It is commonly used to create a filament for Fused Filament Fabrication (FFF) or Fused Deposition Modeling (FDM) 3D printing~\cite{RefWorks:RefID:266-leary2020material}. The general process of extrusion for research or commercial use is shown below in Figure~\ref{fig:literatureReview:generalExtrusionProcess}.

\begin{figure}[h!]
        \centering
        \includegraphics[width=\linewidth]{../figs/literature_review/extrusion/general_extrusion_process.png}
        \caption{Overview of extrusion process. Adapted from~\cite{RefWorks:RefID:266-leary2020material}.}
        \label{fig:literatureReview:generalExtrusionProcess}
\end{figure}

\subsection{Tabletop Extrusion Mechanics\label{sec:literatureReview:extrusion:mechanics}}

The extrusion process is divided into multiple stages including: loading the extrusion material into a hopper, a screw pushing the material through a barrel heated to a specific temperature, the melted material being pushed through a nozzle, and the new filament being cooled and collected. This general process is visualized below in Figure~\ref{fig:literatureReview:extrusionParameters}.

\begin{figure}[h!]
        \centering
        \includegraphics[width=0.5\linewidth]{../figs/literature_review/extrusion/extrusion_parameters_overview.png}
        \caption{Visualization of extrusion parameters~\cite{RefWorks:RefID:309-3devo20253devo}.}
        \label{fig:literatureReview:extrusionParameters}
\end{figure}

Not shown in Figure~\ref{fig:literatureReview:extrusionParameters} is the cooling and collecting process of the extruded filament. The cooling process can be done via forced air convection such as fans pointed at the filament or through water quenching~\cite{RefWorks:RefID:266-leary2020material}.

Extruders for research range in size and price, with the smallest being tabletop extruders. Popular examples of these (from least to most expensive) include the Felfil Evo, Filabot EX6, or 3Devo Filament Maker~\cite{RefWorks:RefID:371-bakhtardesign}. Figure~\ref{fig:literatureReview:tabletopExtruders} shows the extrusion and spooling setup for these various systems.

\begin{figure}[H]
        \centering
        \includegraphics[width=0.5\linewidth]{../figs/literature_review/extrusion/tabletop_extruders.png}
        \caption{Various Lab-Scale Tabletop Extruders. (a) Felfil Evo and Spooler+~\cite{RefWorks:RefID:391-felfil}, (b) Filabot EX6~\cite{RefWorks:RefID:390-filabot}, (c) 3Devo Filament Maker~\cite{RefWorks:RefID:392-3devo}.}
        \label{fig:literatureReview:tabletopExtruders}
\end{figure}

\subsubsection{Important Parameters\label{sec:literatureReview:extrusion:mechanics:parameters}}
There are many parameters that need to be optimized for each material when extruding filament.

\paragraph*{Before Extruding}

Before performing the extrusion, the material preparation and selection are important. Most materials require drying before the extrusion. Materials are either hygroscopic or non-hygroscopic. If a material is hygroscopic, that means it absorbs moisture from the air and thus requires drying to eliminate this moisture. If a material is non-hygroscopic, it does not absorb moisture, but moisture can still accumulate on its surface through condensation. As a result, it is helpful when extruding to dry all materials beforehand~\cite{RefWorks:RefID:395-plastic}.

Additionally, a material's melt flow index (MFI) is a crucial parameter to indicate how well it can be extruded. This property indicates a polymer's flow characteristics or rheological properties in molten state. The MFI is a measure of the extrusion flow in grams per 10 minutes ($\frac{gram}{10 min}$)~\cite{RefWorks:RefID:312-melt}. Using materials with an MFI less than ten is ideal for extrusion~\cite{RefWorks:RefID:309-3devo20253devo}. Additionally, if mixing materials or transitioning from one material to another, it is important to ensure the materials have compatible MFIs, or they likely will not extrude smoothly. A material's MFI can be measured using ASTM standard D1238-23a~\cite{RefWorks:RefID:55-test}.

\paragraph*{While Extruding}
The barrel temperature, crucial for properly melting the extrusion material, can be divided into a single or multiple heat zones. These zone(s) must be optimized to avoid any complications with the extrusion~\cite{RefWorks:RefID:269-2018set}. The melt temperature zone(s) should be adjusted when there are changes to the raw material, excessive moisture, clogging, or changes to the screw speed~\cite{RefWorks:RefID:269-2018set}.

When performing the extrusion, the extrusion screw RPM, cooling fan speed, and spooler pull speed are also all important parameters that must be adjusted~\cite{RefWorks:RefID:393-3devo, RefWorks:RefID:394-2024make}.

\subsection{Possible Issues with Extruding\label{sec:literatureReview:extrusion:issues}}

\paragraph*{Filament Quality}
There are many factors to look for when evaluating filament quality following an extrusion. Issues with filament quality can stem from multiple sources, but likely result from non-optimized extrusion parameters (See Section~\ref{sec:literatureReview:extrusion:mechanics:parameters}). These issues can include: nozzle build-up, bumps or particles in filament, inconsistent filament thickness, low or thin outut, smoke at nozzle, ovular or flat filament, or rough filament surface~\cite{RefWorks:RefID:396-3devotroubleshooting}. An overview of these filament quality issues is shown below in Figure~\ref{fig:literatureReview:extrusion_filament_quality_issues}

\begin{figure}[H]
        \centering
        \includegraphics[width=\linewidth]{../figs/literature_review/extrusion/filament_quality_issues.png}
        \caption{Examples of issues with filament quality when extruding~\cite{RefWorks:RefID:396-3devotroubleshooting}.}
        \label{fig:literatureReview:extrusion_filament_quality_issues}
\end{figure}

\paragraph*{Improper Temperature Zone(s)}
If the temperature zone(s) of the extruder are not optimal for a given material, resulting problems include inhomogeneities in the melt, distortion, issues cooling the filament, low throughput, black spots and specs in the filament, and deterioriation of mechanical properties~\cite{RefWorks:RefID:269-2018set}.

\paragraph*{Grinding or Squeaking Noise}
It's possible to experience a loud grinding or squeaking noise when extruding, resembling metal on metal. This is likely a result of the screw touching the barrel, which is often caused by a particle that is too large moving through the barrel. This large unmelted particle pushes on the screw thereby affecting its alignment and causing the screw to contact the barrel~\cite{RefWorks:RefID:396-3devotroubleshooting}. This issue is shown visually in Figure~\ref{fig:literatureReview:extrusion_grinding_noise}

\begin{figure}[h!]
        \centering
        \includegraphics[width=0.7\linewidth]{../figs/literature_review/extrusion/grinding_noise_cause.png}
        \caption{Cause of Grinding Noise from Large Particle~\cite{RefWorks:RefID:396-3devotroubleshooting}.}
        \label{fig:literatureReview:extrusion_grinding_noise}
\end{figure}

\paragraph*{Clumping at Hopper}

A common issue when extruding is clumping at the hopper, the area where the material enters the extruder. This often occurs when using regrind and variable size materials, or with premature melting if a material has a low melt point. As materials clump together, cohesive structures, ratholing or bridging, in the hopper can form that in turn limit flow rate~\cite{RefWorks:RefID:393-3devo}. This clumping process is shown in Figure~\ref{fig:literatureReview:hopper_clumping}

\begin{figure}[h!]
        \centering
        \includegraphics[width=0.5\linewidth]{../figs/literature_review/extrusion/hopper_clumping.png}
        \caption{Cases of clumping at the hopper. Green shape indicates clumping structure formed~\cite{RefWorks:RefID:393-3devo}.}
        \label{fig:literatureReview:hopper_clumping}
\end{figure}

\paragraph*{Clogging}

In extreme cases, the barrel can clog with material when it has not melted thoroughly. This can lead to pressure buildup in the barrel and unusable filament output. The screw motor current is a helpful indicator of a clog~\cite{RefWorks:RefID:396-3devotroubleshooting}.

\subsection{Extruding Regrind\label{sec:literatureReview:extrusion:regrind}}
% Discuss reprocessing of polymer waste or failed prints.

Significant research has been conducted on the shredding and recycling of 3D printed filament and parts. Predominantly, cost and sustainability are factors that drive these efforts. Different parameters need to be accounted for when extruding recycled/shredded material compared to virgin (unused) material~\cite{RefWorks:RefID:60-ailenei2025study}.

\subsubsection{Machines for Creating Regrind\label{sec:literatureReview:extrusion:regrind:machines}}
% Mention industrial grinders, the Felfil Shredder, and similar tabletop systems.
Dedicated shredders are  required to create regrind from plastic parts or scraps. Like tabletop extruders, these range in functionality and cost~\cite{RefWorks:RefID:218-lee2018development}. These machines can be standalone shredders such as the Felfil Shredder~\cite{RefWorks:RefID:391-felfil} or machines like the 3Devo SHR3D IT that combine other functions like pelletizing or extruding~\cite{RefWorks:RefID:218-lee2018development,RefWorks:RefID:396-3devotroubleshooting}.


\subsubsection{Challenges Extruding Regrind\label{sec:literatureReview:extrusion:regrind:challenges}}
As mentioned in Section~\ref{sec:literatureReview:extrusion:issues}, extruding regrind can lead to inconsistent flow rates and clumping at the hopper. To address this, specific devices have been developed such as the 3Devo Feeder. These are comprised of vibration motors to shake the material and help force it from the hopper into the extrusion barrel. The feeder is shown below in Figure~\ref{fig:literatureReview:3devo_feeder}

\begin{figure}[h!]
        \centering
        \begin{subfigure}[b]{0.48\linewidth}
                \centering
                \includegraphics[width=\textwidth]{../figs/literature_review/extrusion/3devo_feeder_standalone.png}
                \caption{3Devo Feeder (standalone).}
        \end{subfigure}
        \hfill
        \begin{subfigure}[b]{0.48\linewidth}
                \centering
                \includegraphics[width=\textwidth]{../figs/literature_review/extrusion/3devo_feeder_in_use.png}
                \caption{3Devo Feeder in use.}
        \end{subfigure}
        \caption{3Devo Feeder for regrind or powders~\cite{RefWorks:RefID:396-3devotroubleshooting}.}
        \label{fig:literatureReview:3devo_feeder}
\end{figure}

Size selection of regrind is also important, because size homogeneity when extruding influences the quality of the extruded filaments~\cite{RefWorks:RefID:60-ailenei2025study}. It was also found that regrinded filament often exhibited a rough and easily curved surface~\cite{RefWorks:RefID:216-herianto2020recycled}.

\subsubsection{Effects of Re-Extruding Materials\label{sec:literatureReview:extrusion:regrind:effects}}
% Discuss degradation, viscosity changes, and mechanical property shifts.
Using recycled filament can lead to a decrease in mechanical properties such as tensile strength or strain at break~\cite{RefWorks:RefID:218-lee2018development,RefWorks:RefID:64-10}. The results of one study illustrating the changes in mechanical properties of regrind are shown below in Figure~\ref{fig:literatureReview:regrind_changes_in_mechanical_properties}

\begin{figure}[h!]
        \centering
        \includegraphics[width=0.7\linewidth]{../figs/literature_review/extrusion/regrind_changes_in_mechanical_properties.png}
        \caption{The stress-strain curves of tensile testing for (a) virgin PLA and (b) recycled PLA~\cite{RefWorks:RefID:218-lee2018development}.}
        \label{fig:literatureReview:regrind_changes_in_mechanical_properties}
\end{figure}

\subsection{Powder Extrusions\label{sec:literatureReview:extrusion:powder}}

Powder extrusions are often grouped with regrind as they can cause similar challenges and inconsistencies when extruding~\cite{RefWorks:RefID:396-3devotroubleshooting}. For this reason, devices like the 3Devo Feeder are often used for powder extrusions.

When extruding powders, drying materials before extruding is extremely important. Additionally, extra attention must be dedicated to the cooling process to prevent the extruded filament from being too brittle~\cite{RefWorks:RefID:399-2024transforming}.

Overall, limited research and resources exist regarding filament creation by extruding solely powders.

\subsection{Purging an Extruder\label{sec:literatureReview:extrusion:purging}}
% Explain the importance of regular purging to maintain extrusion quality and prevent contamination.
Purging an extruder is an essential part of the extrusion process. It is done to maintain machine health, remove contaminants, and improve the overall extrusion of your materials. A mechanical purge, the most common type of purge, is performed by sending a purging compound through the barrel of the extruder. This often high-viscosity and abrasive polymer pushes remaining material and contaminants out of the barrel~\cite{RefWorks:RefID:405-purging,RefWorks:RefID:407-carrasco2023evaluation}.

\subsubsection{Importance of Regular Purging\label{sec:literatureReview:extrusion:purging:importance}}
% Note 3Devo’s recommendation to purge monthly or whenever powder is used.
% Emphasize cleaning the barrel, removing contaminants, and ensuring uniform extrusion.
Purging an extruder is a crucial and highly recommended part of efficient extruder usage. It is recommended to purge an extruder after every extrusion~\cite{RefWorks:RefID:401-dynapurge,RefWorks:RefID:405-purging}.

This process pushes the remaining residues of polymer that was being extruded and cleans the inside of the extruder to prepare it for subsequent extrusions. This process helps avoid polymer degradation, resin accumulation, and damage to the extruder~\cite{RefWorks:RefID:405-purging}.

Polymer degradation is when the properties of a polymer change due to environmental factors such as heat. If the polymer is left inside the extruder, given its slow cooldown timeline after shutoff, the polymer will be subject to high heat for extended periods of time. This can cause the polymer to break down leading to variable mechanical and rheological properties. Additionally, this broken down material may not flow smoothly when the extruder is powered back on which could lead to clogs inside the barrel~\cite{RefWorks:RefID:405-purging}.

Resin accumulation can occur when material is left inside the barrel. This old material often sticks to the extruder screw, blocking the flow of new material. As a result, resin accumulation can lead to poor extrusions and contamination in subsequent extrusions~\cite{RefWorks:RefID:405-purging}.

Lastly, failing to purge the extruder can damage the equipment over time. Build up of old material can cause extra stress and pressure on the machine's parts~\cite{RefWorks:RefID:405-purging}.

\subsubsection{Purging Procedures\label{sec:literatureReview:extrusion:purging:procedures}}

Purging procedures vary across manufacturers~\cite{RefWorks:RefID:401-dynapurge,RefWorks:RefID:405-purging}, although the general guidelines are outlined in the following steps:

\begin{enumerate}
        \item Transition from current material to purging compound.
        \item When purging compound has fully transitioned, raise temperature above prior material's extrusion range.
        \item Extrude the purging compound at this higher temperature until it is extruding cleanly.
        \item Lower temperature to normal operating range.
        \item Transition back to normal extrusion material.
        \item Shutoff the machine in preparation for the next extrusion.
              \begin{enumerate}
                      \item Some manufacturers recommend shutting down the extruder with purging compound in the barrel. This allows the compound to gather any remaining contaminants as it cools and contracts, then removes these contaminants when the machine is re-run~\cite{RefWorks:RefID:401-dynapurge}.
              \end{enumerate}
\end{enumerate}

\subsubsection{Purging Compounds\label{sec:literatureReview:extrusion:purging:compounds}}
% Compare compound types, composition, and use cases.
Selecting the proper purging compound based on the materials it will be cleaning is an important part of the purging process.

\paragraph*{Material Compatibility}
% Discuss melt flow index and temperature matching to extruded materials.
% Note when harder materials (e.g., HDPE) are required to flush the purging compound.
The type of polymer to be purged, process temperature range and polarity of materials must be evaluated when selecting a purging compound~\cite{RefWorks:RefID:407-carrasco2023evaluation}. The melt flow index (MFI) of a material is often used to ensure a purging compound will react well with the material to be purged~\cite{RefWorks:RefID:401-dynapurge} (See Section~\ref{sec:literatureReview:extrusion:mechanics:parameters} for more information on MFI).

In some cases, selected purging compounds poorly match the properties of the extrusion material, but are used for increased strength or other features. In this case, there may be an intermediate material such as HDPE used to transition from the purging compound to the extrusion material~\cite{RefWorks:RefID:401-dynapurge,RefWorks:RefID:405-purging}.

\paragraph*{Vendors}
There are many vendors that specialize in development of purging compounds, whereas other manufacturers use generic purging compounds. Often, vendors will create a large range of products to account for various use cases~\cite{RefWorks:RefID:407-carrasco2023evaluation}. For example, Dyna-Purge, a purging compound manufacturer, offers at least three compounds that all can purge the same material but offer different levels of abrasiveness~\cite{RefWorks:RefID:400-dynapurge,RefWorks:RefID:403-dynapurge,RefWorks:RefID:404-dynapurge}.
