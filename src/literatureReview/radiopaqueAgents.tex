\section{Radiopaque Agents\label{sec:literatureReview:radiopaque}}
% Discuss why radiopaque agents are used and their impact on imaging visibility.
% Radiopacity section in RefWorks
Radiopaque agents are commonly embedded in polymer-based medical implants. This is due to polymer's inherently low X-ray attenuation as a result of the low atomic numbers of constituent elements (Hydrogen (1), Carbon (6), Nitrogen (7), and Oxygen (8)). If a material has a low X-ray attenuation, it is referred to as radiolucent, when X-rays can pass through with minimal absorption~\cite{RefWorks:RefID:332-emonde2024radiopacity}.

Routine monitoring of implants is important to evaluate their performance, and this monitoring is ideally performed non-invasively through imaging. Through imaging, medical staff can monitor biodegradation and wear, detect malpositioning or migration, and more accurately place implants~\cite{RefWorks:RefID:332-emonde2024radiopacity}.

Metals and heavy elements are common contrast agents, though the threshold levels are low to account for biocompatibility and risk of leaking. The effect metals can have on mechanical properties of polymers also needs to be considered~\cite{RefWorks:RefID:332-emonde2024radiopacity}.

For bioresorbable stents, the leading contrast agent is barium sulfate (BaSO\textsubscript{4}). Though barium sulfate increases the tensile and radial strength of a polymer, it reduces the ductility and elongation at break~\cite{RefWorks:RefID:332-emonde2024radiopacity}.


\subsection{Existing Radiopaque Filaments\label{sec:literatureReview:radiopaque:filaments}}
% Reference materials and data from Radiopaque Materials section in RefWorks.
Commercially available radiopaque filaments are limited, though significant research exists to expand this market gap. Radiopaque filaments are useful for 3D printed imaging phantoms, tissue models, and other medical implants~\cite{RefWorks:RefID:363-savidevelopment,RefWorks:RefID:351-delemeester2024device,RefWorks:RefID:346-özmenradiopaque,RefWorks:RefID:77-hamedani2018threedimensional,RefWorks:RefID:353-cherubini2024production,RefWorks:RefID:352-zekraouiradiopaque}.

Barium sulfate has been used as a contrast agent when developing radiopaque filament due to its low cost and similar attenuation to bone~\cite{RefWorks:RefID:363-savidevelopment,RefWorks:RefID:345-choi2022characterization,RefWorks:RefID:77-hamedani2018threedimensional}. Zinc oxide (ZnO) has also been used due to its low toxicity and eco-friendly processing abilities~\cite{RefWorks:RefID:353-cherubini2024production}. Additional materials used to create radiopaque filament include tungsten (W), bismuth (III) oxide (Bi\textsubscript{2}O\textsubscript{3}), and antimony trioxide (Sb\textsubscript{2}O\textsubscript{3})~\cite{RefWorks:RefID:352-zekraouiradiopaque,RefWorks:RefID:351-delemeester2024device,RefWorks:RefID:346-özmenradiopaque}.
