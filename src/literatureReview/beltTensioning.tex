\section{Belt Tensioning Systems\label{sec:literatureReview:beltTension}}
% Review mechanical belt tensioning designs in extrusion and motion systems.
% Why is belt tensioning important

Belt tensioning is performed to ensure a belt maintains proper tension during operation. A tensioning component is often used to provide adequate tension. A properly tensioned belt is necessary for the efficient, stable, and reliable operation of a timing belt transmission system~\cite{RefWorks:RefID:479-ulsoy1985design,RefWorks:RefID:481-2024commonly}.

Belt tensioning systems must account for instability due to tensioner resonance, belt resonance, tension variations, and belt critical speed~\cite{RefWorks:RefID:479-ulsoy1985design}.

\paragraph*{Relevance to Research}

Belt tensioning was explored to help improve the automatic pellet dispenser and resolve belt slippage issues (see~\fullref{sec:methodology:starveFeeder:systemReDesign:beltTensioningPrototyping})

\subsection{Inside vs Outside Idler Pulley\label{sec:literatureReview:beltTension:idler}}

There are various methods of designing a belt tensioning system. Common methods, illustrated in Figure~\ref{fig:literatureReview:beltTensioning}, include parallel tensioning, medial tensioning, and outer tensioning~\cite{RefWorks:RefID:481-2024commonly}.

\paragraph*{Parallel Tensioning}

Parallel tensioning is achieved by pulling two synchronous belts. It is more compact than other methods and does not require additional redundant mechanisms~\cite{RefWorks:RefID:481-2024commonly}.

\paragraph*{Medial Tensioning}

Medial tensioning utilizes an inside idler pulley that can either be with or without teeth. This extra component can increase costs of the system as well as potential failure points. It also adds space to the system by causing the synchronous belt to protrude~\cite{RefWorks:RefID:481-2024commonly}.

\paragraph*{Outer Tensioning}

Like medial tensioning, outer tensioning utilizes an idler pulley. This pulley sits on the outside of the synchronous belt which can cause wear and tear to the belt surface. This mechanism utilizes sliding and rolling friction~\cite{RefWorks:RefID:481-2024commonly}.

\begin{figure}[H]
        \centering
        \includegraphics[width=\linewidth]{../figs/literature_review/beltTensioning/belt_tensioning_options.png}
        \caption{Common belt tensioning options. Parallel (left), Medial (middle), and Outer (right)~\cite{RefWorks:RefID:481-2024commonly}.}
        \label{fig:literatureReview:beltTensioning}
\end{figure}
