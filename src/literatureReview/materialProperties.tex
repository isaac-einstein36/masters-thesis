\section{Material Properties\label{sec:literatureReview:materials}}
% Review mechanical and physical properties of both biological and polymeric materials.
This section will review existing literature regarding mechanical properties of the biological and polymeric materials evaluated in this thesis.

\subsection{Breast Tissue Properties\label{sec:literatureReview:materials:breast}}
% Summarize adipose and glandular tissue properties relevant to implant design.
Breast tissue is a combination of predominantly adipose and glandular tissue which sits over the pectoralis major muscle as shown below in Figure~\ref{fig:literatureReview:breast_anatomy}~\cite{RefWorks:RefID:231-gefenmechanics}.

\begin{figure}[h!]
        \centering
        \includegraphics[height=0.7\linewidth]{../figs/literature_review/mechanicalProperties/breast_anatomy.png}
        \caption{Anatomical schematic of a human breast~\cite{RefWorks:RefID:231-gefenmechanics}.}
        \label{fig:literatureReview:breast_anatomy}
\end{figure}

An overview of the mechanical properties of the tissue types that comprise breast tissue are shown below in Table~\ref{tab:literatureReview:breast_mechanical_properties}.

\begin{table}[h!]
        \centering
        \caption{Mechanical properties of tissue components of the breast~\cite{RefWorks:RefID:231-gefenmechanics}.}
        \begin{adjustbox}{max width=\textwidth}

                \begin{tabular}{lcc}
                        \hline
                        \textbf{Tissue Type}                           & \textbf{Elastic Modulus [kPa]} & \textbf{Ultimate Strength [MPa]} \\
                        \hline
                        Ribs                                           & 2,000,000--14,000,000          & 100                              \\
                        Pectoralis major/minor (longitudinal, dynamic) & $\sim$30                       & 0.4--0.7                         \\
                        Pectoralis major/minor (transverse, dynamic)   & 1.5--6                         & 0.4--0.7                         \\
                        Pectoralis major/minor (transverse, static)    & 0.75--3.6                      & 0.4--0.7                         \\
                        Pectoralis fascia                              & 100--2000                      & 20--100                          \\
                        Suspensory ligaments                           & 80,000--400,000                & 40                               \\
                        Glandular tissue                               & 7.5--66                        & No data available                \\
                        Adipose                                        & 0.5--25                        & No data available                \\
                        Skin                                           & 200--3000                      & 20                               \\
                        \hline
                \end{tabular}
        \end{adjustbox}
        \label{tab:literatureReview:breast_mechanical_properties}
\end{table}

Additionally, during a mammography, the right and left breast can experience anywhere from 50 to 200 N of compressive force~\cite{RefWorks:RefID:230-poulosbreast}.

Lastly, it was found over 99 studies that the calculated mass-density of the breast is on average 945 $kg/m^3$~\cite{RefWorks:RefID:174-sanchez2016estimating}.

\subsection{Polymer Properties\label{sec:literatureReview:materials:polymers}}
% Compare PLA, PCL, and PLCL mechanical and thermal characteristics.
Mechanical properties for the main polymers used in this thesis, PLA, PCL, and 70/30 PLCL are shown below in Table~\ref{tab:literatureReview:polymer_material_properties}.

\begin{table}[H]
        \centering
        \caption{Mechanical properties of PCL, PLA, and PLCL materials.}
        \label{tab:literatureReview:polymer_material_properties}
        \begin{adjustbox}{max width=\textwidth}
                \begin{threeparttable}
                        \begin{tabular}{l c c c c}
                                \hline
                                \textbf{Material}                                                                                                                                      & \textbf{Tensile Modulus (E\textsubscript{t})} & \textbf{Compressive Modulus (E\textsubscript{comp})} & \textbf{Flexural Modulus (E\textsubscript{flex})} & \textbf{Tensile Yield Strength} \\
                                \hline

                                PCL                                                                                                                                                    &
                                \begin{tabular}{@{}c@{}}440 ± 3 MPa~\cite{RefWorks:RefID:46-ragaert2014bulk}\\417 ± 25 MPa~\cite{RefWorks:RefID:133-staffa2022mechanical}\end{tabular} &
                                455 ± 2 MPa~\cite{RefWorks:RefID:46-ragaert2014bulk}                                                                                                   &
                                414 ± 10 MPa~\cite{RefWorks:RefID:46-ragaert2014bulk}                                                                                                  &
                                \begin{tabular}{@{}c@{}}17.82 ± 0.47 MPa~\cite{RefWorks:RefID:46-ragaert2014bulk}\\14.7 ± 1.3 MPa~\cite{RefWorks:RefID:133-staffa2022mechanical}\end{tabular}                                                                                                                                                                                       \\
                                \hline

                                PLA                                                                                                                                                    &
                                \begin{tabular}{@{}c@{}}3015 ± 86 MPa~\cite{RefWorks:RefID:133-staffa2022mechanical}\\3600 MPa~\cite{RefWorks:RefID:343-2013development}\end{tabular}  &
                                --                                                                                                                                                     &
                                3800 MPa (ID343)                                                                                                                                       &
                                \begin{tabular}{@{}c@{}}55.9 ± 6.5 MPa~\cite{RefWorks:RefID:133-staffa2022mechanical}\\50 MPa~\cite{RefWorks:RefID:343-2013development}\end{tabular}                                                                                                                                                                                                \\
                                \hline

                                PLCL (70/30)                                                                                                                                           &
                                12 ± 1.2 MPa\tnote{1} ~\cite{RefWorks:RefID:31-fernández2012synthesis}                                                                                 &
                                --                                                                                                                                                     & --                                            &
                                \begin{tabular}{@{}c@{}}17.2 ± 0.7 MPa~\cite{RefWorks:RefID:31-fernández2012synthesis}\\16.1 ± 3.2 MPa~\cite{RefWorks:RefID:19-zhang2021synthesis}\end{tabular}                                                                                                                                                                                     \\
                                \hline
                        \end{tabular}

                        \begin{tablenotes}
                                \item[1] Secant modulus measured at 0.2\% strain.
                        \end{tablenotes}
                \end{threeparttable}
        \end{adjustbox}

\end{table}

\subsection{Effects of BaSO\textsubscript{4} Additives\label{sec:literatureReview:materials:BaSO4}}
% Discuss how radiopaque fillers influence modulus, brittleness, and imaging response.
Research has found that increasing BaSO\textsubscript{4} weight percentage in polymers such as PCL can statistically significantly increase the stiffness while decreasing the tensile yield strength. Incorporating BaSO\textsubscript{4} into polymers was found to toughen and reinforce the material simultaneously~\cite{RefWorks:RefID:32-amestoy2021crystallization,RefWorks:RefID:435-yang2015morphologies}.

Figure~\ref{fig:literatureReview:effectsOfBariumSulfate} highlights the effects of incorporating BaSO\textsubscript{4}~\cite{RefWorks:RefID:435-yang2015morphologies}.

\begin{figure}[h!]
        \centering
        \includegraphics[height=0.5\linewidth]{../figs/literature_review/mechanicalProperties/effects_of_barium_sulfate.png}
        \caption{Effects on Elastic Modulus and Tensile Strength of incorporating barium sulfate (BaSO\textsubscript{4})~\cite{RefWorks:RefID:435-yang2015morphologies}.}
        \label{fig:literatureReview:effectsOfBariumSulfate}
\end{figure}
