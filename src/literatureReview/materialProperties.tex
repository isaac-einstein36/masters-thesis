\section{Material Properties\label{sec:literatureReview:materials}}
% Review mechanical and physical properties of both biological and polymeric materials.
This section will review existing literature regarding mechanical properties of the biological and polymeric materials evaluated in this thesis.

\subsection{Breast Tissue Properties\label{sec:literatureReview:materials:breast}}
% Summarize adipose and glandular tissue properties relevant to implant design.
Breast tissue is a combination of predominantly adipose and glandular tissue which sits over the pectoralis major muscle as shown below in Figure~\ref{fig:literatureReview:breast_anatomy}~\cite{RefWorks:RefID:231-gefenmechanics}.

\begin{figure}[h!]
        \centering
        \includegraphics[height=0.7\linewidth]{../figs/literature_review/mechanicalProperties/breast_anatomy.png}
        \caption{Anatomical schematic of a human breast~\cite{RefWorks:RefID:231-gefenmechanics}.}
        \label{fig:literatureReview:breast_anatomy}
\end{figure}

An overview of the mechanical properties of the tissue types that comprise breast tissue are shown below in Table~\ref{tab:literatureReview:breast_mechanical_properties}.

\begin{table}[h!]
        \centering
        \caption{Mechanical properties of tissue components of the breast~\cite{RefWorks:RefID:231-gefenmechanics}.}
        \begin{adjustbox}{max width=\textwidth}

                \begin{tabular}{lcc}
                        \hline
                        \textbf{Tissue Type}                           & \textbf{Elastic Modulus [kPa]} & \textbf{Ultimate Strength [MPa]} \\
                        \hline
                        Ribs                                           & 2,000,000--14,000,000          & 100                              \\
                        Pectoralis major/minor (longitudinal, dynamic) & $\sim$30                       & 0.4--0.7                         \\
                        Pectoralis major/minor (transverse, dynamic)   & 1.5--6                         & 0.4--0.7                         \\
                        Pectoralis major/minor (transverse, static)    & 0.75--3.6                      & 0.4--0.7                         \\
                        Pectoralis fascia                              & 100--2000                      & 20--100                          \\
                        Suspensory ligaments                           & 80,000--400,000                & 40                               \\
                        Glandular tissue                               & 7.5--66                        & No data available                \\
                        Adipose                                        & 0.5--25                        & No data available                \\
                        Skin                                           & 200--3000                      & 20                               \\
                        \hline
                \end{tabular}
        \end{adjustbox}
        \label{tab:literatureReview:breast_mechanical_properties}
\end{table}

Additionally, during a mammography, the right and left breast can experience anywhere from 50 to 200 N of compressive force~\cite{RefWorks:RefID:230-poulosbreast}.

Lastly, it was found over 99 studies that the calculated mass-density of the breast is on average 945 $kg/m^3$~\cite{RefWorks:RefID:174-sanchez2016estimating}.

\subsection{Polymer Properties\label{sec:literatureReview:materials:polymers}}
% Compare PLA, PCL, and PLCL mechanical and thermal characteristics.
Mechanical properties for the main polymers used in this thesis, PLA, PCL, and 70/30 PLCL are shown below in Table~\ref{tab:literatureReview:polymer_material_properties}.

\hl{[Finish Table]}
\begin{table}[h!]
        \centering
        \caption{Material Mechanical Properties for PLA, PCL, and PLCL (70/30)}
        \label{tab:literatureReview:polymer_material_properties}
        \begin{adjustbox}{max width=\textwidth}

                \begin{tabular}{p{2cm} | p{3.5cm} | p{3cm} | p{3cm} | p{3.5cm}}
                        \toprule
                        \textbf{Material} & \textbf{Tensile Modulus ($\mathbf{E_t}$)}                                & \textbf{Compressive Modulus ($\mathbf{E_{comp}}$)} & \textbf{Flexural Modulus ($\mathbf{E_{flex}}$)} & \textbf{Tensile Yield Strength}                                                 \\
                        \midrule
                        PCL               & $440 \pm 3 \text{ MPa (ID46)}$ \newline $417 \pm 25 \text{ MPa (ID133)}$ & $455 \pm 2 \text{ MPa (ID46)}$                     & $414 \pm 10 \text{ MPa (ID46)}$                 & $17.82 \pm 0.47 \text{ MPa (ID46)}$ \newline $14.7 \pm 1.3 \text{ MPa (ID133)}$ \\
                        \midrule
                        PLA               & $3015 \pm 86 \text{ MPa (ID133)}$ \newline $3600 \text{ MPa (ID343)}$    &                                                    & $3800 \text{ MPa (ID343)}$                      & $55.9 \pm 6.5 \text{ MPa (ID133)}$ \newline $50 \text{ MPa (ID434)}$            \\
                        \midrule
                        PLCL (70/30)      & $12 \pm 1.2 \text{ MPa (Secant Modulus at 0.2\% strain) (ID31)}$         &                                                    &                                                 & $17.2 \pm 0.7 \text{ MPa (ID31)}$ \newline $16.1 \pm 3.2 \text{ MPa (ID19)}$    \\
                        \bottomrule
                \end{tabular}
        \end{adjustbox}
\end{table}



\subsection{Effects of BaSO\textsubscript{4} Additives\label{sec:literatureReview:materials:BaSO4}}
% Discuss how radiopaque fillers influence modulus, brittleness, and imaging response.
