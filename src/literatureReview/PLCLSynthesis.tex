\section{Creating PLCL Blends\label{sec:literatureReview:PLCL}}
% Introduce PLCL copolymer—its rationale, applications, and optimal ratio (70/30 PLA/PCL).

PLCL (poly(l-lactide-co-$\epsilon$-caprolactone)) is a copolymer of poly(lactide acid) (PLA) and poly($\epsilon$-caprolactone) (PCL). These polymers are commonly used implantable materials for their biodegradability, non-toxic degradation products, and biocompatibility. Polyester-based polymers have increasingly been researched for use in biomedical applications~\cite{RefWorks:RefID:303-luo2023fabrication,RefWorks:RefID:31-fernández2012synthesis,RefWorks:RefID:19-zhang2021synthesis,RefWorks:RefID:304-jeong2018mechanical}.

\paragraph*{Relevance to Research}

One goal of this research was to create a PLCL filament. This involves combining PLA and PCL pellets and extruding this blend. Various techniques were evaluated for combining pellets prior to extruding and research was also conducted to determine the optimal blend composition (see~\fullref{sec:methodology:extrudingPLCL:pelletExtrusions:combiningMaterials}). This PLCL extrusion was also 3D printed, so printability and material strength of various blends were factors taken into consideration (see~\fullref{sec:methodology:3dPrintingPLCL}).

\subsection{Effects of Co-Polymer Composition\label{sec:literatureReview:PLCL:composition}}
% Detail the effects of different PLA/PCL compositions to help show why 70/30 is the best choice

The composition of PLCL, specifically the ratio of PLA and PCL significantly affect the overall properties of the copolymer~\cite{RefWorks:RefID:303-luo2023fabrication,RefWorks:RefID:19-zhang2021synthesis}.

\subsubsection{Mechanical Properties\label{sec:literatureReview:PLCL:composition:mechanicalProperties}}

PLA exhibits high strength but uncontrollable biodegradability and excessive brittleness. PCL, alternatively, is a softer material that exhibits low strength but high toughness. When combined to create PLCL, PCL incorporation can counteract the brittleness of PLA but decreases overall strength~\cite{RefWorks:RefID:19-zhang2021synthesis,RefWorks:RefID:303-luo2023fabrication}.

Research has been conducted to evaluate the changes in properties based on various PLA/PCL ratios. These studies conclude that elongation at break generally increases with a higher percentage of PCL, and tensile strength and Young's modulus decrease with an increase in PCL~\cite{RefWorks:RefID:31-fernández2012synthesis,RefWorks:RefID:303-luo2023fabrication,RefWorks:RefID:18-v2022assessing}. Some of these results are shown below in Figure~\ref{fig:literatureReview:copolymerRatioProperties}.

\begin{figure}[h!]
        \centering
        \includegraphics[width=\linewidth]{../figs/literature_review/plclSynthesis/copolymer_ratio_mechanical_properties.png}
        \caption{Changes in mechanical properties as PLA/PCL ratios are adjusted. (A,D) Elongation at break, (B,E) Tensile Strength, (C,F) Young's Modulus~\cite{RefWorks:RefID:303-luo2023fabrication}.}
        \label{fig:literatureReview:copolymerRatioProperties}
\end{figure}

Additionally, elastic behavior was only exhibited when PCL composition was high enough such as at least 30\% as illustrated in Figure~\ref{fig:literatureReview:pclForElasticBehavior}~\cite{RefWorks:RefID:18-v2022assessing}.

\begin{figure}[H]
        \centering
        \includegraphics[width=0.5\linewidth]{../figs/literature_review/plclSynthesis/pcl_for_elastic_behavior.png}
        \caption{Changes in elastic behavior with increased PCL percentage~\cite{RefWorks:RefID:18-v2022assessing}.}
        \label{fig:literatureReview:pclForElasticBehavior}
\end{figure}

\subsubsection{Printability\label{sec:literatureReview:PLCL:composition:Printability}}

The ability to 3D print PLCL is also affected by the copolymer makeup. One study compared the printability of 70/30 and 65/35 PLA/PCL. Through various calibration testing, it was found that 70/30 PLA/PCL was the most printable PLCL copolymer~\cite{RefWorks:RefID:18-v2022assessing}. Figure~\ref{fig:literatureReview:copolymerGridTest} helps illustrate these differences in printability.

\begin{figure}[h!]
        \centering
        \includegraphics[width=0.7\linewidth]{../figs/literature_review/plclSynthesis/copolymer_printability_grid_test.png}
        \caption{Evaulating printability of 65/35 (left) and 70/30 (right) PLCL through a grid test~\cite{RefWorks:RefID:18-v2022assessing}.}
        \label{fig:literatureReview:copolymerGridTest}
\end{figure}

\subsection{Lab Synthesis Methods\label{sec:literatureReview:PLCL:labSynthesis}}
In a laboratory setting, PLCL can be formed using ring-opening polymerization (ROP) or combining PLA and PCL with a dissolving agent~\cite{RefWorks:RefID:303-luo2023fabrication,RefWorks:RefID:304-jeong2018mechanical,RefWorks:RefID:253-åkerlund2022effect}. Figure~\ref{fig:literatureReview:plclSynthesisMethods} illustrates common various PLCL synthesis methods.

\subsubsection{Ring-Opening Polymerization (ROP) Synthesis\label{sec:literatureReview:PLCL:labSynthesis:ropSynthesis}}
% Compare ring-opening polymerization (ROP) and block synthesis routes.
In a laboratory setting, PLCL can be combined through ring-opening polymerization (ROP) to create random or block structure PLCL. Both of these synthesis methods require a dedicated laboratory setup and specific expertise~\cite{RefWorks:RefID:303-luo2023fabrication}.

\paragraph*{Random vs Block PLCL}

While both random PLCL (rPLCL) and block PLCL (bPLCL) require ROP synthesis, bPLCL can be made by incorporating a pre-polymerization step of just poly($\epsilon$-caprolactone). Compared to rPLCL, bPLCL preserves the unique crystalline structure of the soft and hard blocks and also allows for fine-tuning of the mechanical and biodegradability properties of the PLCL copolymer~\cite{RefWorks:RefID:303-luo2023fabrication}. The different processes to synthesise rPLCL and bPLCL are shown below in Figure~\ref{fig:literatureReview:randomVsBlockPLCL}.

\begin{figure}[h!]
        \centering
        \includegraphics[width=\linewidth]{../figs/literature_review/plclSynthesis/random_vs_block_synthesis.png}
        \caption{Random (A) vs Block (B) PLCL Synthesis~\cite{RefWorks:RefID:303-luo2023fabrication}.}
        \label{fig:literatureReview:randomVsBlockPLCL}
\end{figure}

\subsubsection{Dissolving Copolymers\label{sec:literatureReview:PLCL:labSynthesis:dissolving}}
% dissolving/solvent-based blending (chloroform or DCM).

Another option for combining PLA and PCL into a PLCL copolymer is by chemically dissolving the materials together. Common solvents used for this are chloroform or dichloromethane (DCM).

To perform this, PLA and PCL are dissolved in DCM or chloroform under agitation for three to four hours~\cite{RefWorks:RefID:304-jeong2018mechanical,RefWorks:RefID:253-åkerlund2022effect}.

\subsection{Blending Techniques\label{sec:literatureReview:PLCL:blending}}

PLCL can also be synthesized by using heat to melt PLA and PCL together. This has been done via specialized equipment for melting, injection molding, or extrusions~\cite{RefWorks:RefID:62-ning2015additive,RefWorks:RefID:302-chen2025structure,RefWorks:RefID:251-fernández‐tena2023highimpact,RefWorks:RefID:254-navarro-baena2016design}. Figure~\ref{fig:literatureReview:plclSynthesisMethods} illustrates common various PLCL synthesis methods.

\subsubsection{Thermal Combination\label{sec:literatureReview:PLCL:blending:thermalCombination}}
% Melting the materials together and pelletizing
% RefID 302 (melted before injection molding)

PLA and PCL can be melt-compounded through equipment such as an RM-200C torque internal mixer. This combines the materials into a PLCL "cake" that can be pelletized~\cite{RefWorks:RefID:302-chen2025structure}.

\subsubsection{Injection Molding\label{sec:literatureReview:PLCL:blending:injectionMolding}}

Injection molding is also a viable method to combine PLA and PCL into PLCL. Some literature employing injection molding have used this technology to fabricate tensile testing specimen~\cite{RefWorks:RefID:302-chen2025structure,RefWorks:RefID:251-fernández‐tena2023highimpact}.

Injection molding alone has been found to poorly mix materials, however, causing some studies to combine PLA and PCL into a mixture before injection molding~\cite{RefWorks:RefID:302-chen2025structure}.

\subsubsection{Extruding Raw Materials Together\label{sec:literatureReview:PLCL:blending:coextrusion}}

PLA and PCL can be combined thermally via an extruder. Either single or twin screw extruders can be used for this operation, although literature gravitates towards using a twin screw extruder for more accurate mixing~\cite{RefWorks:RefID:62-ning2015additive,RefWorks:RefID:254-navarro-baena2016design,RefWorks:RefID:251-fernández‐tena2023highimpact}.

\paragraph*{Single Screw Extruder}

While limited research exists regarding creating PLCL through a single screw extruder, one study did use a single screw extruder to combine a base material (ABS) with a powder filler (carbon fiber)~\cite{RefWorks:RefID:62-ning2015additive}.

This study re-extruded the material to ensure adequate mixing and increase the bulk density, which in turn led to more consistent flow rate when 3D printing. ~\cite{RefWorks:RefID:62-ning2015additive}.

Other research concluded that single screw extruders could carry and combine a binder and filler material, but inhomogeneously~\cite{RefWorks:RefID:363-savidevelopment}.

\paragraph*{Twin Screw Extruder}

Multiple studies utilize a twin screw extruder to combine PLA and PCL in synthesizing PLCL~\cite{RefWorks:RefID:254-navarro-baena2016design,RefWorks:RefID:251-fernández‐tena2023highimpact}. Twin screw extruders, specifically co-rotating extruders, are ideal for mixing multiple materials compared to single screw extruders~\cite{RefWorks:RefID:419-twin}.

\begin{figure}[h!]
        \centering
        \includegraphics[width=\linewidth]{../figs/literature_review/plclSynthesis/plcl_synthesis_options.png}
        \caption{Overview of PLCL synthesis options. (A) laboratory synthesis, (B) Melt-compounding, (C) Injection molding, (D) Extrusion.}
        \label{fig:literatureReview:plclSynthesisMethods}
\end{figure}
