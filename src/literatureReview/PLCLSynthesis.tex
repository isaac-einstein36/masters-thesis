\section{Creating PLCL Blends\label{sec:literatureReview:PLCL}}
% Introduce PLCL copolymer—its rationale, applications, and optimal ratio (70/30 PLA/PCL).

PLCL (poly(l-lactide-co-$\epsilon$-caprolactone)) is a copolymer of poly(lactide acid) (PLA) and poly($\epsilon$-caprolactone) (PCL). These polymers are commonly used implantable materials for their biodegradability, non-toxic degradation products, and biocompatibility. Polyester-based polymers have increasingly been researched for use in biomedical applications~\cite{RefWorks:RefID:303-luo2023fabrication,RefWorks:RefID:31-fernández2012synthesis,RefWorks:RefID:19-zhang2021synthesis,RefWorks:RefID:304-jeong2018mechanical}.

\subsection{Effects of Co-Polymer Composition\label{sec:literatureReview:PLCL:composition}}
% Detail the effects of different PLA/PCL compositions to help show why 70/30 is the best choice

\subsection{Synthesis Methods\label{sec:literatureReview:PLCL:synthesis}}
% Compare ring-opening polymerization (ROP) and block synthesis routes.

\subsection{Blending Techniques\label{sec:literatureReview:PLCL:blending}}
% Discuss melting methods (thermal) and solvent-based blending (chloroform or DCM).

\subsubsection{Injection Molding\label{sec:literatureReview:PLCL:blending:injectionMolding}}
% Explain how injection molding can melt the materials together but doesn't mix them well (maybe the lack of mixing will go in discussion if I can't find papers and am only basing that off of Rachmat's opinion).

\subsubsection{Extruding Raw Materials Together\label{sec:literatureReview:PLCL:blending:coextrusion}}
% Differentiate between single-screw and twin-screw extruders in blend formation.
