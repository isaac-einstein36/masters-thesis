\section{Creating PLCL Blends\label{sec:literatureReview:PLCL}}
% Introduce PLCL copolymer—its rationale, applications, and optimal ratio (70/30 PLA/PCL).

PLCL (poly(l-lactide-co-$\epsilon$-caprolactone)) is a copolymer of poly(lactide acid) (PLA) and poly($\epsilon$-caprolactone) (PCL). These polymers are commonly used implantable materials for their biodegradability, non-toxic degradation products, and biocompatibility. Polyester-based polymers have increasingly been researched for use in biomedical applications~\cite{RefWorks:RefID:303-luo2023fabrication,RefWorks:RefID:31-fernández2012synthesis,RefWorks:RefID:19-zhang2021synthesis,RefWorks:RefID:304-jeong2018mechanical}.

\subsection{Effects of Co-Polymer Composition\label{sec:literatureReview:PLCL:composition}}
% Detail the effects of different PLA/PCL compositions to help show why 70/30 is the best choice

The composition of PLCL, specifically the ratio of PLA and PCL significantly affect the overall properties of the copolymer~\cite{RefWorks:RefID:303-luo2023fabrication,RefWorks:RefID:19-zhang2021synthesis}.

\subsubsection{Mechanical Properties\label{sec:literatureReview:PLCL:composition:mechanicalProperties}}

PLA exhibits high strength but uncontrollable biodegradability and excessive brittleness. PCL, alternatively, is a softer material that exhibits low strength but high toughness. When combined to create PLCL, PCL incorporation can counteract the brittleness of PLA but decreases overall strength~\cite{RefWorks:RefID:19-zhang2021synthesis,RefWorks:RefID:303-luo2023fabrication}.

Research has been conducted to evaluate the changes in properties based on various PLA/PCL ratios. These studies conclude that elongation at break generally increases with a higher percentage of PCL, and tensile strength and Young's modulus decrease with an increase in PCL~\cite{RefWorks:RefID:31-fernández2012synthesis,RefWorks:RefID:303-luo2023fabrication,RefWorks:RefID:18-v2022assessing}. Some of these results are shown below in Figure~\ref{fig:literatureReview:copolymerRatioProperties}.

\begin{figure}[h!]
        \centering
        \includegraphics[width=\linewidth]{../figs/literature_review/plclSynthesis/copolymer_ratio_mechanical_properties.png}
        \caption{Changes in mechanical properties as PLA/PCL ratios are adjusted. (A,D) Elongation at break, (B,E) Tensile Strength, (C,F) Young's Modulus~\cite{RefWorks:RefID:303-luo2023fabrication}.}
        \label{fig:literatureReview:copolymerRatioProperties}
\end{figure}

Additionally, elastic behavior was only exhibited when PCL composition was high enough such as at least 30\% as illustrated in Figure~\ref{fig:literatureReview:pclForElasticBehavior}~\cite{RefWorks:RefID:18-v2022assessing}.

\begin{figure}[H]
        \centering
        \includegraphics[width=0.5\linewidth]{../figs/literature_review/plclSynthesis/pcl_for_elastic_behavior.png}
        \caption{Changes in elastic behavior with increased PCL percentage~\cite{RefWorks:RefID:18-v2022assessing}.}
        \label{fig:literatureReview:pclForElasticBehavior}
\end{figure}

\subsubsection{Printability\label{sec:literatureReview:PLCL:composition:Printability}}

The ability to 3D print PLCL is also affected by the copolymer makeup. One study compared the printability of 70/30 and 65/35 PLA/PCL. Through various calibration testing, it was found that 70/30 PLA/PCL was the most printable PLCL copolymer~\cite{RefWorks:RefID:18-v2022assessing}. Figure~\ref{fig:literatureReview:copolymerGridTest} helps illustrate these differences in printability.

\begin{figure}[h!]
        \centering
        \includegraphics[width=0.7\linewidth]{../figs/literature_review/plclSynthesis/copolymer_printability_grid_test.png}
        \caption{Evaulating printability of 65/35 (left) and 70/30 (right) PLCL through a grid test~\cite{RefWorks:RefID:18-v2022assessing}.}
        \label{fig:literatureReview:copolymerGridTest}
\end{figure}

\subsection{Lab Synthesis Methods\label{sec:literatureReview:PLCL:labSynthesis}}
In a laboratory setting, PLCL can be formed using ring-opening polymerization (ROP) or combining PLA and PCL with a dissolving agent. (\hl{Add citations for ROP and DCM synthesis})

\subsubsection{Ring-Opening Polymerization (ROP) Synthesis\label{sec:literatureReview:PLCL:labSynthesis:ropSynthesis}}
% Compare ring-opening polymerization (ROP) and block synthesis routes.
In a laboratory setting, PLCL can be combined through ring-opening polymerization (ROP) to create random or block structure PLCL. Both of these synthesis methods require a dedicated laboratory setup and expertise~\cite{RefWorks:RefID:303-luo2023fabrication}.

\paragraph*{Random vs Block PLCL}

While both random PLCL (rPLCL) and block PLCL (bPLCL) require ROP synthesis, bPLCL can be made by incorporating a pre-polymerization step of just poly($\epsilon$-caprolactone). Compared to rPLCL, bPLCL preserves the unique crystalline structure of the soft and hard blocks and also allows for fine-tuning of the mechanical and biodegradability properties of the PLCL copolymer~\cite{RefWorks:RefID:303-luo2023fabrication}. The different processes to synthesise rPLCL and bPLCL are shown below in Figure~\ref{fig:literatureReview:randomVsBlockPLCL}.

\begin{figure}[h!]
        \centering
        \includegraphics[width=\linewidth]{../figs/literature_review/plclSynthesis/random_vs_block_synthesis.png}
        \caption{Random (A) vs Block (B) PLCL Synthesis~\cite{RefWorks:RefID:303-luo2023fabrication}.}
        \label{fig:literatureReview:randomVsBlockPLCL}
\end{figure}

\subsubsection{Dissolving Copolymers\label{sec:literatureReview:PLCL:labSynthesis:dissolving}}
% dissolving/solvent-based blending (chloroform or DCM).
% RefID 304, 253

\subsection{Blending Techniques\label{sec:literatureReview:PLCL:blending}}
% Combining PLA and PCL through melting, injection molding, or extrusion
% The carbon fiber article (RefID 62) combines ABS and powder just through extrusion though there's nothing on mixing PLA and PCL solely in an extruder

\subsubsection{Thermal Combination\label{sec:literatureReview:PLCL:blending:thermalCombination}}
% Melting the materials together and pelletizing
% RefID 302 (melted before injection molding)

\subsubsection{Injection Molding\label{sec:literatureReview:PLCL:blending:injectionMolding}}
% Explain how injection molding can melt the materials together but doesn't mix them well (maybe the lack of mixing will go in discussion if I can't find papers and am only basing that off of Rachmat's opinion).
% RefID 302, 251

\subsubsection{Extruding Raw Materials Together\label{sec:literatureReview:PLCL:blending:coextrusion}}
% Differentiate between single-screw and twin-screw extruders in blend formation.
% Single Screw:RefID 62 combined ABS and powder via an extruder and re-extruded to blend better
% Dual Screw: RefID 254, 251
