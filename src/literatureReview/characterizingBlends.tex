\section{Characterizing Polymer Blends\label{sec:literatureReview:characterization}}
% Summarize analytical methods used to verify polymer composition and structure.
When mixing polymers into a single blend, it is important to characterize the morphology, rheological, and mechanical properties of the blend~\cite{RefWorks:RefID:450-matta2014preparation,RefWorks:RefID:444-hassanderelectron}.

There are various methods of characterizing polymer blends including: electron microscopy, spectroscopy such as NMR and FT-IR, and gas chromatography~\cite{RefWorks:RefID:448-feng2021determination,RefWorks:RefID:447-haq2017characterization,RefWorks:RefID:449-grande-tovar2022polycaprolactone,RefWorks:RefID:452-eidelman2004characterization,RefWorks:RefID:442-molineux2023polymers,RefWorks:RefID:444-hassanderelectron,RefWorks:RefID:450-matta2014preparation,RefWorks:RefID:451-negaresh2024polylactic,RefWorks:RefID:453-jordicase,RefWorks:RefID:454-paulsendetermination,RefWorks:RefID:455-mauricio2020determination}.

\subsection{NMR Testing\label{sec:literatureReview:characterization:NMR}}
% Discuss NMR for quantifying PLA/PCL ratio and copolymer composition.
Nuclear Magnetic Resonance (NMR) testing can be useful in determining the percent composition of a polymer blend. This is done by evaluating the integrals of unique peaks on a H NMR spectrum~\cite{RefWorks:RefID:453-jordicase,RefWorks:RefID:454-paulsendetermination,RefWorks:RefID:455-mauricio2020determination,RefWorks:RefID:456-chakrapani2019lowfield}. An example of this spectrum is shown below in Figure~\ref{fig:literatureReview:nmrSpectrum}.

\begin{figure}[h!]
        \centering
        \includegraphics[width=0.7\linewidth]{../figs/literature_review/polymerCharacterization/nmr_spectrum_example.png}
        \caption{Example of H NMR spectrum for composition analysis~\cite{RefWorks:RefID:455-mauricio2020determination}.}
        \label{fig:literatureReview:nmrSpectrum}
\end{figure}

\subsubsection{Powder Preparation\label{sec:literatureReview:characterization:NMR:powderPrep}}
% Compare mortar and pestle, ball milling, and 3D printed filament grinding.
\hl{(Finish this - grinding filament into powder)}

\subsection{FT-IR\label{sec:literatureReview:characterization:ftir}}
% Using FT-IR in polymer chractertization and why it's not as helpful for % composition

\subsection{Microscopy\label{sec:literatureReview:characterization:microscopy}}
% Include optical and scanning electron microscopy for morphological analysis.

\subsection{Gel Permeation Chromatography (GPC)\label{sec:literatureReview:characterization:GPC}}
% Review methods for determining molecular weight distribution and polymer degradation.

\subsection{Characterizing PLA/PCL Blends\label{sec:literatureReview:characterization:pla_pcl}}
% Show papers specifically citing characterization of PLCL or PLA/PCL blends
