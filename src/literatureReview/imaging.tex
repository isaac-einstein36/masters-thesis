\section{Imaging Studies\label{sec:literatureReview:imaging}}
% Summarize imaging studies assessing radiopacity and HU quantification.
Imaging studies are important and often required for the FDA approval of any 510(k) implant device. Imaging studies are important in determining the radiopacity of the device and whether the device impacts imaging needs of surrounding areas~\cite{RefWorks:RefID:433-2023evidentiary}.

\subsection{Interpreting Hounsfield Units\label{sec:literatureReview:imaging:HU}}
Hounsfield units (HU) are a universally used unit in Computer Tomography (CT) imaging. HU are used to quantitatively measure radiodensity, or a tissue/objects ability to absorb X-rays~\cite{RefWorks:RefID:271-hu}.

HU are based on a scale with distilled water being at 0, less dense objects being more negative, and more radiodense objects being more positive. HU imaging is grayscale, and the more positive an object is on the HU scale, the brighter it appears.

% A material's HU is calculated using the linear attenuation coefficients of distilled water (0) and air (-1000) as well as the material's attenuation using the following Equation~\eqref{eq:calculating_HU} below~\cite{RefWorks:RefID:332-emonde2024radiopacity}.

% \begin{equation}
%         HU = 1000 * \frac{\mu}{2}
%         \label{eq:calculating_HU}
% \end{equation}

% Explain how HU correlates to material density and imaging contrast.
% RefID 332 (In Radiopacity, introduction talks about HU)
% RefID 363 Equation 2

\subsection{Test Standards\label{sec:literatureReview:imaging:standards}}
% Mention ASTM F640-20 and similar standards for CT evaluation (Ref ID 342)
% ASTM 5640-20 standard to test radiopacity (332).
\subsection{Prior Studies\label{sec:literatureReview:imaging:priorStudies}}
% Include references such as Ref ID 77 for comparable imaging validation studies.
