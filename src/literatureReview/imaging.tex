\section{Imaging Studies\label{sec:literatureReview:imaging}}
% Summarize imaging studies assessing radiopacity and HU quantification.
Imaging studies are important and often required for the FDA approval of any 510(k) implant device. Imaging studies are important in determining the radiopacity of the device and whether the device impacts imaging needs of surrounding areas~\cite{RefWorks:RefID:433-2023evidentiary}.

\subsection{Interpreting Hounsfield Units\label{sec:literatureReview:imaging:HU}}
Hounsfield units (HU) are a universally used unit in Computer Tomography (CT) imaging. HU are used to quantitatively measure radiodensity, or a tissue/objects ability to absorb X-rays~\cite{RefWorks:RefID:271-banookhu}.

HU are based on a scale with distilled water being at 0, less dense objects being more negative, and more radiodense objects being more positive. HU imaging is grayscale, and the more positive an object is on the HU scale, the brighter it appears.

A material's HU is calculated using the linear attenuation coefficients of distilled water (0) and air (-1000) as well as the material's attenuation using the following Equation~\eqref{eq:calculating_HU} below~\cite{RefWorks:RefID:332-emonde2024radiopacity,RefWorks:RefID:363-savidevelopment}.

\begin{equation}
        HU = 1000 \times \frac{\mu-\mu_{water}}{\mu_{air}-\mu_{water}}
        \label{eq:calculating_HU}
\end{equation}

Figure~\ref{fig:literatureReview:HU_scale} illustrates where various hard and soft tissues in the human body fall on the HU scale~\cite{RefWorks:RefID:332-emonde2024radiopacity}.

\begin{figure}[h!]
        \centering
        \includegraphics[height=0.5\linewidth]{../figs/literature_review/imaging/hu_scale.png}
        \caption{Various soft and hard tissues in the human body on the HU scale~\cite{RefWorks:RefID:332-emonde2024radiopacity}.}
        \label{fig:literatureReview:HU_scale}
\end{figure}

\subsection{Test Standards\label{sec:literatureReview:imaging:standards}}

Radiopacity is measured both quantitatively and qualitatively through analyzing shape and texture~\cite{RefWorks:RefID:271-banookhu}.

Quantitatively, ASTM F640-20, Test Method for Determining Radiopacity for Medical Use, is often followed~\cite{RefWorks:RefID:342-test}.

\subsection{Prior Studies\label{sec:literatureReview:imaging:priorStudies}}

Multiple papers and studies exist that perform similar radiopacity testing to that in this thesis~\cite{RefWorks:RefID:77-hamedani2018threedimensional,RefWorks:RefID:363-savidevelopment,RefWorks:RefID:346-özmenradiopaque}.
