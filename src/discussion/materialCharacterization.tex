\section{PLCL Material Characterization Discussion\label{sec:discussion:materialCharacterization}}

\subsection{GPC Testing Discussion\label{sec:discussion:materialCharacterization:gpcTesting}}

The initial PLCL GPC testing established a methodology for performing subsequent tests. At the time of initial testing, it was anticipated that all extrusions would be powder-based and therefore multiple batches of Nomisma PLCL powder would have to be ordered. Since then, extrusions have shifted to pellet-based (see ~\ref{sec:methodology:extrudingPLCL:pelletExtrusions}). While the sample preparation will have to be adjusted to dissolve pellets rather than powders, this methodology can still be applied to future PLCL synthesis.

The calculated molecular weight through GPC testing was roughly 77,700 g/mol. This differs from the molecular weight of roughly 120,000 g/mol quoted by Nomisma~\cite{RefWorks:RefID:387-nomisma}. This discrepancy reaffirms the importance of confirming a materials molecular weight before begining extrusion research. Additionally, if multiple batches of PLCL powder were extruded over time, their precicely calculated molecular weight could be used as a variable to assess and predict that batch's extrudability.

\subsection{NMR Testing Discussion\label{sec:discussion:materialCharacterization:nmrTesting}}

\paragraph*{Future Testing}
The NMR testing established all necessary training to conduct NMR tests in the future. With this methodology established, future PLCL extrusions can be tested to veerify their percent composition. This will be especially useful in comparing first-pass PLCL to second or third-pass PLCL and see if the mixture is more incorporated with each re-extrusion.

\paragraph*{Material Preparation}
It was anticipated that the material would have to be in powder form to dissolve quickly based on conversations with CBC staff (see ~\fullref{sec:literatureReview:characterization:NMR:powderPrep} for more information on dissolution rate).

Fortunately, the solid samples dissolved fully and rapidly in deuterated chloroform. This means that pulverization methods will not be necessary to conduct NMR testing (see~\fullref{sec:literatureReview:characterization:NMR:powderPrep} for information on possible pulverization methods).

\paragraph*{Results}

The calculated result of 65/35 percent composition PLA/PCL is near the ideal 70/30 composition. This corroborates the material preparation of mixing by hand (see~\fullref{sec:discussion:extrudingPLCL:pelletExtrusions:combiningMaterials}). In addition to the mixing method, it is possible that the PLA/PCL blend was not exactly a 70/30 mixture prior to being poured in the hopper due to estimated weighing. Thus, it is possible the NMR calculated percent compositions are more accurate than initially appearing if the blend was in fact closer to 65/35 than 70/30.

\paragraph*{Key Takeaways}

The NMR testing was successful in establishing a methodology for quantitatively determining percent composition of the PLA/PCL blend. Initial testing revealed an estimated 70/30 blend was in fact roughly 65/35\% PLA/PCL.

It was also found through this testing that solid materials could rapidly dissolve in deuterated chloroform, so methods of pulverizing filament/materials does not need to be further explored.

Lastly, this testing can be a useful tool in evaluating percent composition across multiple extrusion passes to determine the optimal number of extrusions required to create a well-mixed blend.
