\section{3D Printing BaSO\textsubscript{4} Discussion\label{sec:discussion:3dPrintingBaSO4}}

\subsection{Optimizing Printing Parameters Discussion\label{sec:discussion:3dPrintingBaSO4:printingParameters}}

It was found that using the Bambu Marble presets without modifications resulted in adequate printing across most barium sulfate filaments. The only change that had to be made was the nozzle printing temperature. By raising the second layer nozzle temperature from 220\textcelsius ~to 230\textcelsius, any clogging or underextrusion was resolved.

For 10\% barium sulfate-percentage filament, clogging was still observed in the first layer, so the first layer nozzle temperature was raised from 220\textcelsius ~to 225\textcelsius. These printing parameters, when prepared using the sample methodology described (see~\fullref{sec:methodology:extrudingBaSO4:samplePrep}), printed reliably across multiple extrusion batches of filament.

\subsection{Next Steps\label{sec:discussion:3dPrintingBaSO4:nextSteps}}

Once printability of the extruded filaments was established, various 3D printed samples could be created to conduct imaging and mechanical testing. This would help evaluate the effects on material properties or incorporating barium sulfate in a material to in turn optimize the barium sulfate percentage required in the PLCL implant.
