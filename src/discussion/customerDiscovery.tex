\section{Customer Discovery Discussion\label{sec:discussion:customerDiscovery}}

\subsection{Lumpectomy Observation Discussion\label{sec:discussion:customerDiscovery:lumpectomyObservation}}

\paragraph*{Implantation Time}

One key finding through observing lumpectomy procedures is that the implantation time of competitor products such as fiducial clips is very quick (a matter of seconds). This is something that surgeons appreciate by not having to spend excessive time arranging or readjusting the tumor markers.

Thus, the final design must prioritize implantation speed to be competitive with existing devices.

\paragraph*{Variability of Tumor Geometry}

Initially, it was assumed that the imaging-based tumor geometry determined prior to surgery would closely match the excised tumor cavity following a lumpectomy surgery. These observations highlighted the fact that the amount and areas of excised tissue is highly volatile and can change multiple times over the course of a lumpectomy surgery.

As a result, the final device must be highly customizable and malleable to conform to fluid tumor cavity geometries.

\paragraph*{Differing Opinions on Device Volume}

Based on challenges and patient discomfort experienced with the Biozorb marker (see~\fullref{sec:literatureReview:currentMethods:challengeswithcurrentdevicesandmethods:biozorb}), it was assumed that a thick device should be avoided for a thin low-profile implant. Many surgeons agreed with this to avoid the cosmetic downsides of a potentially protruding device. However, some surgeons appreciated the volume of devices like Biozorb that could fill and replace excised tissue.

Based on this feedback, various concepts were brainstormed to make a thin device that could also provide filling and support for a now emptier area of the body.

\paragraph*{Key Takeaways}

Shadowing lumpectomy procedures were incredibly helpful for better understanding the surgery and how competitor devices are used in practice. These observations illustrated to the research team the importance of implantation time and an adjustable final geometry of the implant. Additionally, an interest in a 3D implant to replace space from excised tissue was established.

All these findings directly impact the overall device design, and creating a device with end-user needs in mind improves the chances of adoption when the device is brought to market.
