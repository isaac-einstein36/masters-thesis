\section{External Labs Discussion\label{sec:discussion:externalLabs}}

\subsection{CDME Extrusion Recycling\label{sec:discussion:externalLabs:CDME}}

The Center for Design and Manufacturing Excellence (CDME) was predominantly utilized to perform mechanical testing. Tensile and flexural testing was performed to evaluate the effects of incorporating barium sulfate into a base material (see ~\fullref{sec:methodology:effectsOfBaSO4}).

It was discovered through various connections across CDME, though, that the research lab was exploring recycling 3D printed parts. This involved gathering 3D printed scraps around campus, shredding the scraps, and re-extruding the regrind into usable filament.

While the end goal of this research, 3D printing sustainability, differs from the extrusion research discussed in this thesis, creating a custom material to 3D print with, the process of shredding and re-extruding filament is similar. Therefore, a tour was scheduled to better understand CDME's shredding and extrusion capabilities.

\subsubsection{Shredding Capabilities\label{sec:discussion:externalLabs:CDME:shredding}}

This research team has utilized a small tabletop shredding system, the Felfil Shredder. The CDME lab, alternatively, owns large-scale plastic shredders. These machines can be fed larger quantities of plastic and create a smaller, more uniform output than the Felfil Shredder. Example output of this shredder is shown below in Figure~\ref{fig:discussion:externalLabs:cdme:shredderOutput}.

\begin{figure}[h!]
        \centering
        \includegraphics[width=0.5\linewidth]{../figs/discussion/externalLabs/cdme/shredder_output.png}
        \caption{Output of CDME shredder.}
        \label{fig:discussion:externalLabs:cdme:shredderOutput}
\end{figure}

\subsubsection{Future Collaboration\label{sec:discussion:externalLabs:CDME:futureCollaboration}}

Because of CDME's more robust shredding equipment, it would benefit the research team to utilize their resources for future regrind research as opposed to using the existing Felfil Shredder system. The Felfil Shredder produced non-uniform regrind which may have contributed to inconsistent filament thickness (see~\fullref{sec:methodology:extrudingPLCL:felfilSystem:felfilShredder}). Using a more uniform regrind as input may improve results of extruding reground filament to better refine the filament thickness of extruded PLCL.
