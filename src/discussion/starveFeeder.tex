\section{Starve Feeder Discussion\label{sec:discussion:starveFeeder}}

\hl{Discussion: With this setup, one designated computer could be used for all aspects of extruding -- monitoring the extruder functionality and controlling the pellet dispensing into the extruder. Additionally, the keypad and LCD could be removed from the system which substantially reduced clutter.}

1. Accuracy is bad at low outputs (0.5g) but comparable and fine at higher outputs
2. Accuracy/tolerance doesn't have to be too tight since it's just filling the barrel opening and the amount and time between pours can be adjusted in real time. The amount needed is qualitative based on the extruder feed rate rather than quantitative.
3. The belt tensioner in combination with the rubber belt and stronger gears works well and resolved all belt slipping issues. These changes were made simultaneously, so it's unclear if one fixed the slippage more than the other.
4. The feeder as a whole makes the process more reliable and easier. A researcher no longer has to monitor the hopper constantly over the course of a multi-hour extrusion. It's also more reliable to provide a constant repeated output so that the hopper never gets too full or too empty.
5. This kind of system should be employed in tabletop extrusion for materials with low melt temperatures that are prone to melting prematurely.
