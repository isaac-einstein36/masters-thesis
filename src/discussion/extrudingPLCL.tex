\section{Extruding PLCL Discussion\label{discussion:extrudingPLCL}}

\subsection{Powder Extrusion Discussion\label{sec:discussion:extrudingPLCL:powderExtrusion}}

While this extrusion was successful in that filament was extruded, the extruded filament was unusable. When evaluating filament quality from a 3D printable perspective, there are numerous factors that must be considered (see~\fullref{sec:literatureReview:extrusion:issues} for a detailed explanation of these factors).

\paragraph*{Filament Output}
The filament was highly brittle and could snap easily. Although the powder was dried prior to extrusion and allowed to cool in air, brittleness was still a significant concern. This is the third extruder the team has used for powder extrusions (see~\fullref{sec:introduction:priorWork:otherTeamWork:plclExtrusions} for more information) and all three extruders led to highly brittle output. As a result, it is hypothesized at this point that a fully powder-based extrusion may create unusably brittle filament.

Additionally, the output was thin and experienced a variable flow rate. The Devostick pusher was used to address feed rate variability, but this did not adequately solve the issue. The variable feed rate and Devostick also led to concerning spikes in current as shown in~\autoref{fig:results:extrudingPLCL:powderExtrusion:currentOutput}.

Lastly, the filament had air bubbles inside indicative of temperature being too high. If the temperature is too high, the powder is likely degrading into a gas~\cite{RefWorks:RefID:396-3devotroubleshooting}. These air bubbles are shown below in~\autoref{fig:results:extrudingPLCL:powderExtrusion:airBubbles}

\begin{figure}[h!]
        \centering
        \includegraphics[width=0.3\linewidth]{../figs/discussion/plclExtrusions/powderExtrusion/air_bubbles.png}
        \caption{Performing PLCL powder extrusion. Measuring powder (left) and pouring into extruder (right).}
        \label{fig:results:extrudingPLCL:powderExtrusion:airBubbles}
\end{figure}

\paragraph*{Extrusion Process}

In addition to the filament output, there were concerns with the extrusion process as a whole. It was noted that the powder formed clumps (bridging) at the hopper. Even with the Devostick pusher, these clumps could not be consistently broken up which contributed to the variable flow rate. It was determined based on feedback from 3Devo customer support that this clumping is likely due to solely using powders and the heat zone temperatures being too high.

Also, the process of extruding powder was found to be chaotic and messy as well as inefficient. It was difficult to keep the powder from dispersing when transferring from the initial container to the hopper. This led to dirtying of equipment and wasted raw material.

\paragraph*{Key Takeaways}
% order feeder, powders will always be brittle
Key takeaways from this extrusion experiment are that: a 3Devo Feeder~\cite{RefWorks:RefID:310-3devofeeder} should be ordered to address clumping and variable flow rate, heat zones should be decreased to address clumping, cleaning supplies should be ordered to address powder dispersion, and alternative extrusion routes should be explored to address inherent brittleness of powder extrusions.

Because the Nomisma PLCL powder is expensive and has a long lead time, the team decided to pause powder extrusion research until the 3Devo Feeder arrived to avoid potentially wasting raw material.

\subsection{Pellet Extrusions Discussion\label{sec:discussion:extrudingPLCL:pelletExtrusions}}

\subsubsection{Combining Materials Discussion\label{sec:discussion:extrudingPLCL:pelletExtrusions:combiningMaterials}}

As mentioned in ~\fullref{sec:methodology:extrudingPLCL:pelletExtrusions:combiningMaterials}, various methods were explored to combine PLA and PCL pellets into a homogenous mixture.

\paragraph*{College of Pharmacy}

It was determined that the College of Pharmacy did not have adequate equipment to assist in mixing PLA and PCL pellets. Any mixing equipment they were able to provide was meant for small-scale (a few gram) mixing.

\paragraph*{Injection Molding}

While injection molding was deemed a possible option for combining these materials, it was eventually ruled out based on limitations of the equipment and the research team.

Discussions with professors at Ohio State University as well as a review of relevant literature made it apparent that injection molders could combine materials but would not reliably mix the materials homogeneously.

Various faculty were also hesitant to recommend injection molding given the steep learning curve and high setup time required to operate the equipment.

\paragraph*{Chemically Combining Materials}

After discussing the option of chemically combining the pellets with the PSRF (see~\autoref{sec:methodology:externalLabs:polymersLab}) using chloroform or DCM as a dissolving agent, the team was told this could not be done. For safety purposes, the PSRF no longer carries DCM or chloroform and is not allowed to work with these materials in most cases.

The final output of thermally combining the materials on a hotplate, as shown in~\autoref{fig:methodology:extrudingPLCL:meltingOnHotplate}, was too large to easily shred into extrudable pellets. Melting these materials in a shallower and wider dish may have resulted in a more shreddable output. However, this could not be done easily on a hotplate for the scale of materials needed for extruding. As a result, this rudimentary method of combining materials was eliminated.

\paragraph*{3D Printable Mixing Systems}

The research team decided that 3D printable tumbler systems were usable for mixing materials but that it may require more time than expected to fully assemble these systems. The team felt that if dedicated mixing equipment was required, this would be the fastest and cheapest approach. However, the team wanted to first see if they needed dedicated mixing equipment at all before spending time and resources in building these systems.

\paragraph*{Discussions with ISE Department}

Discussions with the ISE department were helpful in planning out multiple best and worst-case options for mixing materials given available equipment, time, and material constraints.

Dr. Mulyana was confident that dedicated mixing equipment was not required and that the pellets could be manually mixed in a container prior to extrusion. Due to a single screw extruder's inherently poor mixing capabilities, the limiting factor in creating a homogenous output would be the extruder rather than the degree of homogeneity of the pre-extrusion mixture.

To address this bottleneck, Dr. Mulyana offered the team uses a twin screw extruder in the ISE department. While this equipment would mix materials well, it would require high volumes of raw materials to run and time to learn how to use this equipment. It was decided that a twin screw extruder would be a viable backup option if using a single screw extruder did not mix materials well enough.

\paragraph*{Key Takeaways}

Many options were explored to combine PLA and PCL pellets into a homogenous mixture prior to extruding. Based on time, cost, safety, and anticipated limitations of equipment, injection molding, combining materials in a lab setting, and 3D printing a mixing system were no longer explored.

After consulting with Dr. Mulyana, the team opted to start by mixing pellets manually in a jar, leaving other methods for later consideration.

\subsubsection{Initial Pellet Extrusion Discussion\label{sec:discussion:extrudingPLCL:pelletExtrusions:initialPelletExtrusion}}

The team was surprised that this extrusion led to clumping at the hopper given the use of the Feeder and pellets rather than powders. The expected outcome of using the Feeder was that its vibrations would keep pellets from being stagnant long enough to start bridging.

Conversations with 3Devo customer support following this extrusion revealed that the Feeder can not be used with more than one homogenous material. When a blend is poured into the hopper, the Feeder's vibrations will cause solid segregation, causing one material to sink to the bottom and the other to rise to the top. This was directly seen in the extrusion with PLA moving to the bottom of the hopper and PCL moving to the top.

It was expected that temperatures were low enough to prevent premature melting of material. However, PCL pellets appeared to melt and bridge in the hopper. 3Devo customer support believed this is because the metal hopper is not insulated, so material sitting in the hopper for extended periods of time can be subject to additional heat. Because PCL has a lower melting point than PLA, the PLA pellets did not melt while sitting in the hopper but the PCL pellets began to.

As a result, it was recommended to explore PCL's workable extrusion temperature range to determine how high temperatures could be set for this material. Ideally the upper end of the PCL extrusion temperature range would overlap with the lower end of the PLA extrusion temperature range. This overlap would be the ideal extrusion temperature for the blend of materials.

Starve feeding, or pouring material directly into the barrel rather than letting it sit in the hopper, was suggested to prevent PCL from prematurely melting in the uninsulated metal hopper. It was also hypothesized that insulating the hopper could help prevent this premature melting.

\paragraph*{Key Takeaways}

This extrusion helped drive research forward by highlighting new concerns with multi-material pellet extrusions. It was found that the Feeder could not be used on blends of materials and PCL could not sit in the metal hopper for extended periods of time without beginning to soften and melt.

Based on this extrusion, the next steps were to extrude PCL alone and determine its working extrusion temperature range.

\subsubsection{PCL Temperature Study Discussion\label{sec:discussion:extrudingPLCL:pelletExtrusions:pclTempStudy}}

Based on the rise in current at 60-80\textcelsius, ~3Devo customer support hypothesized that the PCL was too solid and forming a clog inside the barrel. As a result, PCL should not be extruded below 80\textcelsius.

No issues were seen with feeding or filament output between 80\textcelsius ~and 150\textcelsius. Premature melting began at 160\textcelsius. Filament output was still consistent at these higher temperatures, but starve feeding had to be more closely monitored to prevent bridging at the hopper.

Figure~\ref{fig:discussion:extrudingPLCL:pelletExtrusions:pclTempRange} illustrates the working temperature range for PCL pellets based on this temperature study.

\begin{figure}[h!]
        \centering
        \includegraphics[width=\linewidth]{../figs/discussion/plclExtrusions/pelletExtrusions/pcl_extrusion_temperature_range.png}
        \caption{PCL working extrusion temperature range.}
        \label{fig:discussion:extrudingPLCL:pelletExtrusions:pclTempRange}
\end{figure}

Based on this extrusion study, PCL should ideally be extruded between 80 and 150\textcelsius, ~but can be extruded as high as 180\textcelsius ~as long as starve feeding is closely monitored.

\subsubsection{Manual Starve Feeding Discussion\label{sec:discussion:extrudingPLCL:pelletExtrusions:manualStarveFeeding}}

Given the susceptibility of PCL to clog at temperatures above 160\textcelsius, ~starve feeding must be closely monitored when extruding at these temperatures.  When performing all starve feeding manually, the researcher is unable to move away from the extruder due to the short time between starve feeding sessions. It was found that starve feeding was required roughly every two minutes or less and at precise amounts. Because extrusions often last multiple hours, manual starve feeding is not an efficient or sustainable process.

As a result, an automatic starve feeding system should be developed to assist with starve feeding.
