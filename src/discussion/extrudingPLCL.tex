\section{Extruding PLCL\label{discussion:extrudingPLCL}}

\subsection{Powder Extrusion\label{discussion:extrudingPLCL:powderExtrusion}}

While this extrusion was successful in that filament was extruded, the extruded filament was unusable. When evaluating filament quality from a 3D printable perspective, there are numerous factors that must be considered (see~\fullref{sec:literatureReview:extrusion:issues} for a detailed explanation of these factors).

\paragraph*{Filament Output}
The filament was highly brittle and could snap easily. Although the powder was dried prior to extrusion and allowed to cool in air, brittleness was still a significant concern. This is the third extruder the team has used for powder extrusions (see~\fullref{sec:introduction:priorWork:otherTeamWork:plclExtrusions} for more information) and all three extruders led to highly brittle output. As a result, it is hypothesized at this point that a fully powder-based extrusion may create unusably brittle filament.

Additionally, the output was thin and experienced a variable flow rate. The Devostick pusher was used to address feed rate variability, but this did not adequately solve the issue. The variable feed rate and Devostick also led to concerning spikes in current as shown in~\autoref{fig:results:extrudingPLCL:powderExtrusion:currentOutput}.

Lastly, the filament had air bubbles inside indicative of temperature being too high. If the temperature is too high, the powder is likely degrading into a gas~\cite{RefWorks:RefID:396-3devotroubleshooting}. These air bubbles are shown below in~\autoref{fig:results:extrudingPLCL:powderExtrusion:airBubbles}

\begin{figure}[h!]
        \centering
        \includegraphics[width=0.3\linewidth]{../figs/discussion/plclExtrusions/powderExtrusion/air_bubbles.png}
        \caption{Performing PLCL powder extrusion. Measuring powder (left) and pouring into extruder (right).}
        \label{fig:results:extrudingPLCL:powderExtrusion:airBubbles}
\end{figure}

\paragraph*{Extrusion Process}

In addition to the filament output, there were concerns with the extrusion process as a whole. It was noted that the powder formed clumps (bridging) at the hopper. Even with the Devostick pusher, these clumps could not be consistently broken up which contributed to the variable flow rate. It was determined based on feedback from 3Devo customer support that this clumping is likely due to solely using powders and the heat zone temperatures being too high.

Also, the process of extruding powder was found to be chaotic and messy as well as inefficient. It was difficult to keep the powder from dispersing when transferring from the initial container to the hopper. This led to dirtying of equipment and wasted raw material.

\paragraph*{Key Takeaways}
% order feeder, powders will always be brittle
Key takeaways from this extrusion experiment are that: a 3Devo Feeder~\cite{RefWorks:RefID:310-3devofeeder} should be ordered to address clumping and variable flow rate, heat zones should be decreased to address clumping, cleaning supplies should be ordered to address powder dispersion, and alternative extrusion routes should be explored to address inherent brittleness of powder extrusions.

Because the Nomisma PLCL powder is expensive and has a long lead time, the team decided to pause powder extrusion research until the 3Devo Feeder arrived to avoid potentially wasting raw material.
