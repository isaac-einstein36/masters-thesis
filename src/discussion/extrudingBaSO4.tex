\section{Extruding BaSO\textsubscript{4} Discussion\label{sec:discussion:extrudingBaSO4}}

\paragraph*{Filament Quality}

All extruded filaments (0\% to 10\%) appeared to be well-mixed with barium sulfate. After crushing the barium sulfate prior to extruding, the output filament was homogeonously white in color rather than clear with specks of barium sulfate. This clear mixing is shown in Figure~\ref{fig:discussion:extrudingBaSO4:mixedFilament} which illustrates the difference between 0\% and 7.5\% barium sulfate.

\begin{figure}[h!]
        \centering
        \includegraphics[width=\linewidth]{../figs/discussion/baSO4Extrusions/comparing_mixing.png}
        \caption{0\% barium sulfate filament (top) compared to 7.5\% barium sulfate. Notice color change between filaments.}
        \label{fig:discussion:extrudingBaSO4:mixedFilament}
\end{figure}

\paragraph*{Filament Thickness}

All extrusions maintained proper filament thickness as shown in Table~\ref{tab:results:extrudingBaSO4:filamentThicknesses}. Because of this and the filament quality, there was no need to adjust the extrusion parameters such as heat zones or extruder RPM.

\paragraph*{Key Takeaways}

When prepared following the methodology described (see~\fullref{sec:methodology:extrudingBaSO4:samplePrep}), all samples were extruded well following the PLA presets. Average filament thickness was within the ideal bounds of $1.65mm$ to $1.85mm$.
