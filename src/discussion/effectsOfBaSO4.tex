\section{Evaluating Effects of Barium Sulfate Discussion\label{sec:discussion:effectsOfBaSO4}}

\subsection{Imaging Study Discussion\label{sec:discussion:effectsOfBaSO4:imagingStudy}}

\paragraph*{Successful Incorporation of Barium Sulfate}

The imaging study illustrated that barium sulfate was successfully incorporated into the PLA filament. As the barium sulfate concentration increased from 0\% to 10\%, the Hounsfield Units (HU) steadily increased as well. This proves that the barium sulfate was properly mixed into the composite filament. Thus, this is a usable methodology of extruding and printing barium sulfate with a raw material.

Based on the growing radiopacity of each sample and comparison to Bonlecule filament, this imaging study illustrated that the established filament creation methodology results in similarly incorporated material as a commercially available filament.

\paragraph*{Radiopacity of Bonlecule Filament}

Based on the imaging study results, the bonlecule filament had a radiopacity between that of 0\% and 2.5\% barium sulfate. Interpolating the results and assuming radiopacity is roughly linearly proportional to percentage of barium sulfate, Equation~\eqref{eq:interpolatingBaSO4Results} can be used to estimate how much barium sulfate concentration the bonlecule filament compares to.

\begin{equation}
        BaSO_4(estimated\%) = \frac{HU_{Bonlecule}}{HU_{2.5\%} - HU_{0\%}} * (2.5\% - 0\%)
        \label{eq:interpolatingBaSO4Results}
\end{equation}

Using Equation~\eqref{eq:interpolatingBaSO4Results} and the results of the imaging study, it is estimated that the Bonlecule filament would measure similarly to a 0.513\% concentration of barium sulfate.

This metric is helpful to know as the optimal concentration of barium sulfate is determined. If more than 0.513\% barium sulfate is required for radiological imaging needs, then the Bonlecule filament cannot be utilized for further testing and development. However, if 0.513\% or roughly 133HU is adequate from an imaging perspective, then this filament can be utilized in lieu of extruding a custom filament.

Mechanical testing was also performed to evaluate how the Bonlecule filament compares to barium sulfate-infused PLA filament.

\paragraph*{Required Amount of Barium Sulfate}

According to radiation oncology staff at OSUCCC, an HU difference of roughly 300 is recommended for ideal contrast between the implanted device and the surrounding tissue. According to OSUCCC, breast tissue has an HU of roughly -100, but a seroma formation has the density of water (0 HU). Thus, a 300HU contrast would require an implant exhibiting 200 to 300 HU.

Based on the results of this imaging study, the Bonlecule filament at 130 HU may be too low, while the 2.5\% barium sulfate filament at 685 HU may be too high and provide excessive contrast. Thus, future testing should be conducted using concentrations between 0\% and 2.5\% barium sulfate. Additionally, however, a thinner sample will lower the apparent HU value, according to OSUCCC staff, so testing should also be conducted with samples that are closer to the projected thickness of the final implant.

\paragraph*{Key Takeaways}

The imaging study illustrated that barium sulfate was successfully incorporated into the PLA filament. Also, as the concentration of barium sulfate increased, the relative HU value of the samples increased as well.

The Bonlecule filament exhibited an HU value between 0\% and 2.5\% barium sulfate samples.

Additionally, based on these HU values, a concentration of barium sulfate between 0\% and 2.5\% should be used to provide adequate contrast between the implant and surrounding tissue. This may need to be adjusted, however, for thinner imaging samples.

\subsection{Tensile Testing Discussion\label{sec:discussion:effectsOfBaSO4:tensileTesting}}

\subsubsection{Type V Specimen Tensile Testing Discussion\label{sec:discussion:effectsOfBaSO4:tensileTesting:typeV}}

\paragraph*{Tensile Strength}

It was expected, based on relevant literature, that the tensile strength should decrease with an increased concentration of barium sulfate~\cite{RefWorks:RefID:32-amestoy2021crystallization,RefWorks:RefID:435-yang2015morphologies}. However, the tensile strength remained relatively similar across all sample groups as the correlation coefficient between the data was 0.1876 which indicates a weak linear relationship.

\paragraph*{Need for Additional Testing}

It was hypothesized by staff at CDME that the test specimen were too small for their Instron to measure accurate data from. As a result, it was recommended that Type IV specimen be tested to utilize a strain gauge and ensure proper gripping of test specimens.

Based on these insights, i`t is possible that inaccuracies arose from the sample being too small for the machine which could have contributed to the unexpected insignificant affects of barium sulfate on tensile strength.

\paragraph*{Key Takeaways}

Because elastic modulus could not be recorded and CDME staff expressed concerns over test specimen size, Type IV samples should be 3D printed and tested.

\subsubsection{Type IV Tensile Testing Discussion\label{sec:discussion:effectsOfBaSO4:tensileTesting:typeIV}}

The results of the Type IV tensile testing did not appear to exhibit a pattern or show a direct effect of barium sulfate on tensile strength or elastic modulus.

\paragraph*{Possible Limitations}
Minimal samples were run for this testing. The testing standard recommends at least five test specimens be used, but due to material limitations, only 2 to 5 were used for each sample group. It is possible that this limited sample size contributed to skewed data. To rule this out, future testing should be performed with an adequate sample size and printing with multiple sections of the extruded filament to ensure all portions of the extrusion are being tested.

It is also possible that the print orientation (flat on the print bed) impacted the overall results of this testing. As detailed in~\fullref{sec:literatureReview:printing:optimalParameters:orientation}, the print orientation has a direct impact on overall component strength. To test this, multiple test specimens can be printed across various print orientations.

\paragraph*{Key Takeaways}

The results of this experiment did not align with existing research which found an inverse relationship between barium sulfate concentration and tensile strength~\cite{RefWorks:RefID:32-amestoy2021crystallization,RefWorks:RefID:435-yang2015morphologies}.

There are multiple possible confounding variables that could lead to this difference in findings including limited test samples and differences in print orientation of samples. Future testing can be conducted to either reaffirm or challenge the findings of this experiment.

\subsection{Flexural Testing Discussion\label{sec:discussion:effectsOfBaSO4:flexuralTesting}}
