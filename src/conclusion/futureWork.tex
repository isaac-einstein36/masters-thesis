\section{Future Work\label{sec:conclusion:futureWork}}

\subsection{Extruding PLCL\label{sec:conclusion:futureWork:extrudingPLCL}}

This research made significant progress in the extrusion process of PLCL. However, the filament thickness is still too variable for reliable 3D printing. Thus, the filament thickness needs to be further refined. Methods for this include using shredded first-pass PLCL filament or optimizing the extruding parameters of first-pass PLCL filament.

\subsection{Creating Regrind\label{sec:conclusion:futureWork:creatingRegrind}}

Regrind, or shredded filament, was created using a Felfil Shredder. Given the increased functionality of the equipment, it may be helpful to partner with the Center for Design and Manufacturing Excellence (CDME) to utilize their large-scale shredding equipment.

\subsection{Incorporating Barium Sulfate\label{sec:conclusion:futureWork:incorporatingBaSO4}}

Barium sulfate was successfully incorporated into PLA via extrusion. The long-term goal of this research is to incorporate barium sulfate with PLCL. This will likely require the further optimization of extruding PLCL before barium sulfate can be added. However, a methodology for incorporating barium sulfate in a base material through extrusion has been established.

\subsection{Customer Discovery\label{sec:conclusion:futureWork:customerDiscovery}}

Multiple surgical oncologists have been observed and interviewed to gather insights into design requirements for this implantable device. Interviews with radiation oncologists were unable to be arranged, but discussing the imaging capabilities of the device with radiation oncologists will provide a necessary perspective as these stakeholders will use the device for radiation therapy treatment planning. In the future, radiation oncologists should be observed and/or interviewed to better understand the imaging perspectives needed for development of this device.

\subsection{Testing Custom PLCL Filament\label{sec:conclusion:futureWork:testingPLCL}}

Because custom PLCL filament was not able to be reliably 3D printed due to variability in filament thickness, mechanical testing could not be performed on this material. In the future, it will be helpful to perform tensile and flexural testing on custom-made PLCL filament. These results can be compared to Lattice Medical PLCL filament~\cite{RefWorks:RefID:42-latticemedical} to validate the custom-made filament with commercially available filament. These results will also serve as a benchmark for how the final device will behave and how incorporating barium sulfate may affect the device material properties.

\subsection{Blend Characterization\label{sec:conclusion:futureWork:blendCharacterization}}

Given available time and resources, NMR and GPC testing were conducted to help characterize various PLCL blends. In the future, additional testing such as scanning electron microscopy (SEM), transmission electron microscopy (TEM) and differential scanning calorimetry (DSC) will ideally be performed to better assess the properties and mixture level of custom-made polymer blends (see~\fullref{sec:literatureReview:characterization}).
