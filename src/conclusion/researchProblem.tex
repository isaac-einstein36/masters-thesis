\section{Review of Research Problem\label{sec:conclusion:researchProblem}}

The goal of this research was to help progress the development of a biodegradable continuously radiopaque implant to improve tumor cavity marking for radiation therapy treatment planning following a lumpectomy procedure.

A lumpectomy procedure is often followed by radiation therapy to the tumor bed to kill any stray cancer cells and prevent cancer recurrence. Once the tumor is removed, however, it is more difficult to accurately mark the area that contained cancerous cells.

Current methods for marking the tumor bed include seroma identification, fiducial markers, or other implants. These methods, however, all have various shortcomings which establishes the need for a biodegradable continuously radiopaque implant.

Prior team research identified Poly(L-lactide-co-$\varepsilon$-caprolactone)(PLCL) as a suitable biodegradable base material and barium sulfate as a suitable radiopaque agent. This thesis aimed to refine the manufacturing process of PLCL to create a 3D printable filament for future implant design research.
