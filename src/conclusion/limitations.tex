\section{Limitations\label{sec:conclusion:limitations}}

\subsection{Tensile Testing\label{sec:conclusion:limitations:tensileTesting}}

The tensile testing performed yielded an insignificant relationship between barium sulfate concentration and yield strength and elastic modulus. This disagrees with existing literature which claims barium sulfate concentration and yield strength are inversely related.

It is possible that a limited number of samples contributed to these inconclusive findings.

\subsection{Shredding Filament\label{sec:conclusion:limitations:shreddingFilament}}

The research team had access to a Felfil Shredder, a small-scale tabletop shredder. With this equipment, the team was unable to create uniformly sized regrind for re-extrusion. It is possible that with more robust equipment, this regrind could be made more uniformly which may impact the extrudability of reground filament.

\subsection{NMR Testing\label{sec:conclusion:limitations:nmrTesting}}

NMR testing was performed to characterize the percent composition of a PLA/PCL blend. One test was performed using a single piece of reground PLCL filament. Thus, these results may not be indicative of the entire output filament and additional tests should be run across the full length of filament to determine the average percent composition.

Additionally, it was determined after completing analysis that the default proton NMR method used does not provide the fullest quantitative results. In the future, a longer quantitative proton method should be used to eliminate potential uncertainties in results.
