\section{Summary of Key Findings\label{sec:conclusion:keyFindings}}

\subsection{Extruding PLCL\label{sec:conclusion:keyFindings:extrudingPLCL}}

It was found that performing solely powder-based extrusions result in substantial challenges and inefficiencies from a processing perspective and often lead to an overly brittle filament.

PLCL was extruded using a pellet-based blend of its co-polymers (PLA and PCL). NMR testing was performed to verify the percent composition of the PLA/PCL blend.

An automatic pellet dispenser was developed and refined to control the release of material into the extruder and prevent premature melting of material. The use of this pellet dispenser made PLCL extrusion possible, although filament thickness was outside of necessary tolerances for 3D printing. PLCL filament was extruded on a Felfil Evo system with an average thickness of 1.75mm, although the variability of thickness caused clogging when 3D printing.

Shredding and re-extruding first-pass PLCL filament was also explored as a method of controlling output filament thickness. It was determined that the shredding process needed to be refined further to allow for consistent extrusion flow of reground filament.

\subsection{Incorporation of Barium Sulfate\label{sec:conclusion:keyFindings:incorporatingBaSO4}}

Barium sulfate was successfully incorporated into PLA to create composite filaments of 0 to 10\% barium sulfate. These materials were tested to evaluate the effects of barium sulfate on material properties.

Tensile testing indicated an unclear relationship between barium sulfate concentration and yield strength and elastic modulus.

Flexural testing indicated a direct relationship between barium sulfate concentration and flexural modulus.

Imaging testing was also performed to evaluate the effects of barium sulfate on radiopacity. It was found that 2.5\% barium sulfate concentration likely provides adequate contrast between this implant and surrounding tissue for radiation therapy planning. Additionally, increasing the concentration of barium sulfate in an extruded filament directly relates to a higher radiopacity of 3D printed material.

\subsection{Purging Extruder\label{sec:conclusion:keyFindings:purgingExtruder}}

Multiple purging compounds were evaluated to clean an extruder following PLA/PCL extrusions. An optimal purging compound, Dyna-Purge K, was chosen based on its cleaning abilities, and minimal required time and materials for purging.
