Breast cancer is the second leading cause of cancer-related deaths in U.S. women and accounts for roughly 30\% of new cancer cases each year in U.S. women~\cite{RefWorks:RefID:36-2021breast,RefWorks:RefID:150-2025breast}. A common treatment small or early-stage breast cancer is a lumpectomy procedure in which the cancerous tissue and a margin of surrounding healthy tissue is removed. This procedure is often followed by targeted radiation to kill any stray cancer cells and prevent recurrence~\cite{RefWorks:RefID:165-czajka2023breast,RefWorks:RefID:159-depolo2024radiation,RefWorks:RefID:157-thomasscience,RefWorks:RefID:198-jiao2024interobserver,RefWorks:RefID:375-joosten2013evaluation}.

Current methods of delineating the tumor bed for radiation planning have limitations~\cite{RefWorks:RefID:25-acree2022review}. A proposed solution is a 3D printed biodegradable continuously radiopaque implant created from Poly(l-lactide-co-$\varepsilon$-caprolactone) (PLCL) with barium sulfate (BaSO$_4$) as a radiopaque contrast agent~\cite{RefWorks:RefID:371-bakhtardesign,RefWorks:RefID:372-krakovskytumor,RefWorks:RefID:370-einsteinisaac}.

This thesis describes development of an extrusion methodology for producing 3D printable PLCL filament and examines the incorporation of BaSO$_4$ in PLA, a copolymer of PLCL, and its effects on material properties.

PLCL was extruded from blended PLA and PCL pellets using a novel pellet-dispensing system. PLA with BaSO$_4$ concentrations of 0\% to 10\% were extruded and 3D printed. Imaging testing showed that 2.5\% or less BaSO$_4$ concentration provides adequate imaging contrast. Mechanical testing revealed a direct relationship between BaSO$_4$ concentration and flexural modulus, and an unclear effect on yield strength and elastic modulus.

This work establishes groundwork for future development of manufacturing methods and device requirements for the PLCL-based tumor bed marking implant.
