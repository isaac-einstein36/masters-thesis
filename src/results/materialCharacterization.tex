\section{PLCL Material Characterization Results\label{sec:results:materialCharacterization}}

\subsection{GPC Testing Results\label{sec:results:materialCharacterization:gpcTesting}}

The full curve and integration region of the GPC testing are shown below in Figure~\ref{fig:results:materialCharacterization:gpcCurve}.

\begin{figure}[H]
        \centering
        \includegraphics[width=\linewidth]{../figs/results/materialCharacterization/gpc_curve.png}
        \caption{Full GPC curve (A) and integration region (B).}
        \label{fig:results:materialCharacterization:gpcCurve}
\end{figure}

At the peak of the integration region, the x-value (logM) is 4.8906. Using Equation~\eqref{eq:logConversion}, this can be converted to molecular weight. The molecular weight following Equation~\eqref{eq:logConversion} is 77,732 g/mol.

\begin{equation}
        M(\frac{g}{mol}) = 10^{logM}
        \label{eq:logConversion}
\end{equation}

Discussion of this testing can be found in~\fullref{sec:discussion:materialCharacterization:gpcTesting}.

\subsection{NMR Testing Results\label{sec:results:materialCharacterization:nmrTesting}}

The NMR spectra for PLA, PCL, and the PLA/PCL blend are shown below in Figure~\ref{fig:results:materialCharacterization:nmrSpectra}.

\begin{figure}[H]
        \centering
        \includegraphics[width=\linewidth]{../figs/results/materialCharacterization/nmr_spectra.png}
        \caption{NMR spectra. PLA (A), PCL (B), and PLA/PCL Blend (C).}
        \label{fig:results:materialCharacterization:nmrSpectra}
\end{figure}

The areas of the 5.15 ppm peak and 2.30 ppm peak were 58126.57 and 31409.37. Using Equation~\eqref{eq:nmrPercentComp}, this results in a 64.92\% composition of PLA and thus a 35.08\% composition of PCL in the ideally 70/30 PLA/PCL blend.

Discussion of this testing can be found in~\fullref{sec:discussion:materialCharacterization:nmrTesting}.
