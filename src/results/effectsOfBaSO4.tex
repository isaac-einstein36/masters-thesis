\section{Evaluating Effects of Barium Sulfate Results\label{sec:results:effectsOfBaSO4}}

\subsection{Imaging Study Results\label{sec:results:effectsOfBaSO4:imagingStudy}}

Resultant images of the imaging study are shown below in Figure~\ref{fig:results:effectsOfBaSO4:imagingStudy}. Two batches of 2.5\% barium sulfate were imaged due to issues printing the first batch. Only the well-printed samples are shown below.

\begin{figure}[h!]
        \centering
        \includegraphics[width=\linewidth]{../figs/results/imagingStudy/imaging_study_screenshots.png}
        \caption{Results of imaging study. (A) 0\% BaSO\textsubscript{4}, (B) 2.5\% BaSO\textsubscript{4}, (C) 5.0\% BaSO\textsubscript{4}, (D) 7.5\% BaSO\textsubscript{4}, (E) 10.0\% BaSO\textsubscript{4}, (F) Bonlecule filament.}
        \label{fig:results:effectsOfBaSO4:imagingStudy}
\end{figure}

A distribution and average of the Hounsfield Units (HU) across each sample group are shown in Figure~\ref{fig:results:effectsOfBaSO4:imagingStudyDistribution} and Figure~\ref{fig:results:effectsOfBaSO4:imagingStudyAverage} respectively.

\begin{figure}[h!]
        \centering
        \includegraphics[width=0.7\linewidth]{../figs/results/imagingStudy/imaging_study_hu_distribution.png}
        \caption{Distribution of HU across sample groups.}
        \label{fig:results:effectsOfBaSO4:imagingStudyDistribution}
\end{figure}

\begin{figure}[h!]
        \centering
        \includegraphics[width=0.7\linewidth]{../figs/results/imagingStudy/imaging_study_hu_average.png}
        \caption{Average HU across sample groups.}
        \label{fig:results:effectsOfBaSO4:imagingStudyAverage}
\end{figure}

\subsection{Tensile Testing Results\label{sec:results:effectsOfBaSO4:tensileTesting}}

\subsubsection{Type V Specimen Tensile Testing Results\label{sec:results:effectsOfBaSO4:tensileTesting:typeV}}

Strain measurements could not be accurately recorded because a small enough strain gage to fit the Type V sample was not available. Thus, a stress vs strain curve was not generated for this testing.

The tensile strength at yield, however, was calculated following Equation~\eqref{eq:tensileStrength}. Results of the tensile strength across all sample groups is shown below in Table~\ref{tab:results:effectsOfBaSO4:tensileTest:tensileStrength} and Figure~\ref{fig:results:effectsOfBaSO4:tensileTest:tensileStrength}.

\begin{table}[h!]
        \centering
        \caption{Average Tensile Strength for Each Sample Group}
        \label{tab:results:effectsOfBaSO4:tensileTest:tensileStrength}
        \begin{tabular}{l c}
                \hline
                \textbf{Sample Group} & \textbf{Average Tensile Strength (MPa)} \\
                \hline
                0\% BaSO$_4$          & 69.00                                   \\
                2.5\% BaSO$_4$        & 60.36                                   \\
                5\% BaSO$_4$          & 60.08                                   \\
                7.5\% BaSO$_4$        & 64.82                                   \\
                10\% BaSO$_4$         & 61.62                                   \\
                Bonlecule             & 58.30                                   \\
                \hline
        \end{tabular}
\end{table}

\begin{figure}[h!]
        \centering
        \includegraphics[width=\linewidth]{../figs/results/mechanicalTesting/tensileTesting/type_v_tensile_strength.png}
        \caption{Distribution of tensile strength for Type V specimens (A). Average tensile strength (B).}
        \label{fig:results:effectsOfBaSO4:tensileTest:tensileStrength}
\end{figure}

Discussion of these results can be found in~\fullref{sec:discussion:effectsOfBaSO4:tensileTesting:typeV}.

\subsubsection{Type IV Tensile Testing Results\label{sec:results:effectsOfBaSO4:tensileTesting:typeIV}}

Table~\ref{tab:results:effectsOfBaSO4:tensileTesting:typeIV} summarizes the results of the Type IV tensile testing.

\begin{table}[h!]
        \centering
        \caption{Type IV tensile strength and elastic modulus for all sample groups.}
        \label{tab:results:effectsOfBaSO4:tensileTesting:typeIV}
        \begin{tabular}{l c c c}
                \hline
                \textbf{Sample Group} & \textbf{Number of Samples} & \textbf{Tensile Strength (MPa)} & \textbf{Elastic Modulus (MPa)} \\
                \hline
                0\% BaSO$_4$          & 5                          & 49.8                            & 2806                           \\
                5\% BaSO$_4$          & 4                          & 53.075                          & 3192.5                         \\
                10\% BaSO$_4$         & 2                          & 43.7                            & 3085                           \\
                \hline
        \end{tabular}
\end{table}

Discussion of these results can be found in~\autoref{sec:discussion:effectsOfBaSO4:tensileTesting:typeIV}.

\subsection{Flexural Testing Results\label{sec:results:effectsOfBaSO4:flexuralTesting}}

All tests were conducted, and flexural modulus was calculated automatically by the Mark-10 software. Table~\ref{tab:results:effectsOfBaSO4:flexuralTesting} summarizes the flexural modulus for each sample group.

\begin{table}[h!]
        \centering
        \caption{Average flexural modulus by sample group.}
        \label{tab:results:effectsOfBaSO4:flexuralTesting}
        \begin{tabular}{l c}
                \hline
                \textbf{Sample Group} & \textbf{Flexural Modulus (MPa)} \\
                \hline
                0\% BaSO$_4$          & 1415.798                        \\
                5\% BaSO$_4$          & 1500.238                        \\
                10\% BaSO$_4$         & 1710.963333                     \\
                \hline
        \end{tabular}
\end{table}

Figure~\ref{fig:results:effectsOfBaSO4:flexuralTesting:resultsSummary} illustrates the distribution of and average flexural modulus across all sample groups.

\begin{figure}[h!]
        \centering
        \includegraphics[width=\linewidth]{../figs/results/mechanicalTesting/flexuralTesting/flexural_testing_results.png}
        \caption{Distribution of flexural modulus (A) and average flexural modulus across sample groups (B).}
        \label{fig:results:effectsOfBaSO4:flexuralTesting:resultsSummary}
\end{figure}

Discussion of these results can be found in Section~\ref{sec:discussion:effectsOfBaSO4:flexuralTesting}.
